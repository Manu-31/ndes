%========================================================================
%
%========================================================================
\section{Tutorial}
\label{section:user-tuto}

%------------------------------------------------------------------------
\subsection{Introduction}

   Congratulations! You are ready to enter the fabulous world of
{\sc ndes}, \ldots
 
    It's basically an artfully embellished library with numerous bugs and ready to help you (or not) to your own network simulator!

    The implementation of a simulator is done by writing a
C program using this library.


%------------------------------------------------------------------------
\subsection{Installing}

\begin{verbatim}
 $ git clone https://github.com/Manu-31/ndes.git
 $ cd ndes
 $ make
 $ make install
\end{verbatim}

   Do not worry, it does not install anything yet! This does
as copying a library (static) in a dedicated directory.

    We can also compile some test programs:

\begin{verbatim}
 $ make tests-bin
\end{verbatim}
%$
    And why not, to run it:

\begin{verbatim}
 $ make tests
\end{verbatim}
%$

   Obviously, the {\tt OK} messages don't prove us anything! Error messages tell us that there are problems \ldots {} as we were expected.

%------------------------------------------------------------------------
\subsection{My first simulation : the file M/M/1 !}

   The source file (and its Makefile, because we respect our client) is found in the directory {\tt example/tutorial-1}.

   In {\sc ndes}, the system will be modeled by a source, followed by a queue, downstream of which it exists a monitoring server followed by a sink. I leave you to draw a picture and I will publish here the prettiest one!

%........................................................................
\subsubsection{Creating a simulator}

   Before any action, we create the simulator in the following way:

\begin{verbatim}
#include <motsim.h>

...

   /* Create a simulator */
   motSim_create();
\end{verbatim}

%........................................................................
\subsubsection{Creating a sink}

   A sink is an object that receives messages without complaining and
which destroys them instantly. It is very simple to create a sink:

\begin{verbatim}
#include <pdu-sink.h>

...

   struct PDUSink_t       * sink; // Déclaration d'un puits

   ...

   /* Create a sink */
   sink = PDUSink_create();
\end{verbatim}

%........................................................................
%
%........................................................................
\subsubsection{Creating a server}
   
   As we do not want to do anything intelligent with our server, we will use a generic server like this:


\begin{verbatim}
#include <srv-gen.h>

...

   struct srvGen_t        * serveur; // Déclaration d'un serveur générique

...

   /* Create a server */
   serveur = srvGen_create(sink, (processPDU_t)PDUSink_processPDU);
\end{verbatim}

   The creation of the server is somewhat more complicated than that of
sinks. The reason is that the server firstly needs to know who to send
clients ({\sc pdu}s for {\sc ndes}). It must therefore say what function they apply (here {\tt PDUSink\_processPDU} a specific function sink) and how an
entity applies this function (our single sink, here). To know the function, it must be sought in the description of the target entity. In our example, the function {\tt PDUSink\_processPDU} is described in the section dedicated to the sinks.

    More details about these functions are found in the section
\ref{section:architecture} which describes the overall architecture.

    As we want a M/M/1 queue, we need to tell our server that its processing time is an exponential parameter {\tt mu}.

  
\begin{verbatim}
   float   mu = 10.0; // parameter of the server

   ...

   /* Set server's parameters */
   srvGen_setServiceTime(serveur, serviceTimeExp, mu);
\end{verbatim}

   Generic servers are described more specifically in the
section \ref{section:srv_gen}.

%........................................................................
%
%........................................................................
\subsubsection{Creating a file}
 
   A file allows to store the {\sc pdu}s in transit. We can create a file 
in the following way: 
  

\begin{verbatim}
   struct filePDU_t       * filePDU; // Declaration of a file pdu

    ...

   /* Create a file pdu */
   filePDU = filePDU_create(serveur, (processPDU_t)srvGen_processPDU);
\end{verbatim}

   Without further add, the files are not bounded. They
are precisely described in section \ref{section:file}. The
two parameters of the function to create the file have the same role as those of the function to create the server.
 
%........................................................................
%
%........................................................................
\subsubsection{Creating a source}

   We will use an object of {\sc ndes} whose role is to produce {\sc pdu}s. But before that, we need to create another object that indicates the time when it produces them: this is a "time generator". Time generators are described in section \ref{section:date_generator}.

    We want a poisonn source, in conclusion an exponential generator:

\begin{verbatim}
#include <date-generator.h>

...

   struct dateGenerator_t * dateGenExp; // Un générateur de dates
   float   lambda = 5.0 ; // Intensité du processus d'arrivée

   ...

   /* Create a data generator */
   dateGenExp = dateGenerator_createExp(lambda);
\end{verbatim}

   And now we can create our source:

\begin{verbatim}
#include <pdu-source.h>

...

   struct PDUSource_t     * sourcePDU;  // Une source

   ...

   sourcePDU = PDUSource_create(dateGenExp, 
				filePDU,
				(processPDU_t)filePDU_processPDU);
\end{verbatim}
  
   The first parameter is the object which allows him to determine
the date of sending. The following two are similar to those passed
when creating the file and the server.
   
%........................................................................
%
%........................................................................
\subsubsection{The implementation of probes}

   
   Yes, I know, the title is a little scary, but it will be simple. 
When you want to run a simulator, it is generaly hoped to get results.
   In {\sc ndes}, they will be collected by a specific tool, the probe(the sensor).

   Different types of sensors are described in section \ref{section:sondes}.
   Here we only use exhaustive probes. For each object described above, the list of available probes is provided.   

   We declare and initialize them like this:   

\begin{verbatim}
#include <probe.h>

...

   struct probe_t         * sejProbe, * iaProbe, * srvProbe; // Les sondes

...

   /* A sensor for inter-arrival time */
   iaProbe = probe_createExhaustive();
   dateGenerator_addInterArrivalProbe(dateGenExp, iaProbe);

   /* A sensor for journey time */
   sejProbe = probe_createExhaustive();
   filePDU_addSejournProbe(filePDU, sejProbe);

   /* A sensor for service time */
   srvProbe = probe_createExhaustive();
   srvGen_addServiceProbe(serveur, srvProbe);
\end{verbatim}

%........................................................................
%
%........................................................................
\subsubsection{Launch of the simulation}
   
   
   That's it, we're finally ready to start the
simulation. We must activate the desired entities at the appropriate
time. Here there is only a single source to activate, and we wish
to do it from the beginning of the simulation:


\begin{verbatim}
   /* We activate de source */
   PDUSource_start(sourcePDU);
\end{verbatim}

   We are now launching the simulation. We'll make it last
100 seconds:

\begin{verbatim}
   /* The simulation lasts 100 s = 100 000ms */
   motSim_runUntil(100000.0);
\end{verbatim}

   We can then display some internal settings of the simulator:

\begin{verbatim}
   motSim_printStatus();
\end{verbatim}

   And that's it!

%........................................................................
%
%........................................................................
\subsubsection{Showing the results}
  
   Now that our simulation is complete, we certainly
want to see the result. We use for this the functions
provided by the probes, for example:  


\begin{verbatim}
   /* Print scalar results */
   printf("%d packets in a file\n",
	  filePDU_length(filePDU));
   printf("Average time of journey = %f\n",
	  probe_mean(sejProbe));
   printf("Average inter-arival   = %f (1/lambda = %f)\n",
	  probe_mean(iaProbe), 1.0/lambda);
   printf("Average time of service = %f (1/mu     = %f)\n",
	  probe_mean(srvProbe), 1.0/mu);
\end{verbatim}

   Great, isn't it? No! There's something more \ldots

%........................................................................
%
%........................................................................
\subsubsection{Drawing the curves}

   
   For better results, we use(from the simulator) a display {\em gnuplot}. We have here at least two interesting plot curves, so we will write a
function for that:

\begin{verbatim}

/*
 * The results will be displayed in a graphbar
 * with nbBar bars with the name "name"
 */
void tracer(struct probe_t * pr, char * name, int nbBar)
{
   struct probe_t   * gb;
   struct gnuplot_t * gp;

   /* The source for the graphbar */
   gb = probe_createGraphBar(probe_min(pr), probe_max(pr), nbBar);

   /* We convert the sensor into a parameter for graphbar */
   probe_exhaustiveToGraphBar(pr, gb);

   /* We name it */
   probe_setName(gb, name);

   /* We initialise the gnuplot */
   gp = gnuplot_create();

   /* We set the range of the graph */
   gnuplot_setXRange(gp, probe_min(gb), probe_max(gb)/2.0);

   /* We display it */
   gnuplot_displayProbe(gp, WITH_BOXES, gb);
}
\end{verbatim}

   The use of {\em gnuplot} is described in the section
\ref{section:gnuplot}.

   Please do not forget to add a short break at the end of our main program, if not it stops and kills the child processes, and therefore the display {\rm gnuplot} dissapears.

%........................................................................
%
%........................................................................
\subsubsection{The use of our first simulator}

   It only remains for us to compile our program and to launch it. The {\tt examples/tutorial-1} also contains a makefile, but firstly, go and take a
look at the includes and the library. Use the {\tt Makefile} like this:

\begin{verbatim}
 $ cd examples/tutorial-1
 $ make
 $ ./mm1
[MOTSI] Date = 99999.846555
[MOTSI] Events : 998243 created (3 m + 998240 r)/998242 freed
[MOTSI] Simulated events : 998243 in, 998242 out, 1 pr.
[MOTSI] PDU : 998242 created (27 m + 998215 r)/998242 freed
[MOTSI] Total malloc'ed memory : 25169976 bytes
[MOTSI] Realtime duration : 1 sec
0 paquets restant dans  la file
Temps moyen de sejour dans la file = 0.061008
Interarive moyenne     = 0.200352 (1/lambda = 0.200000)
Temps de service moyen = 0.100176 (1/mu     = 0.100000)
*** ^C pour finir ;-)
 $
\end{verbatim}

%------------------------------------------------------------------------
%
%------------------------------------------------------------------------
\subsection{The second simulation : another M/M/1 file !}

   The first example is nice, but it would be better
if the processing time depends on the size of the customer (as
packets whose transmission time depends on the size!)

   This is what we will do in this second tutorial. The files are located in the directory {\tt example/tutorial-2}.

   I'll just comment the differences of this tutorial with the first one.

   A small cosmetic change is made in the definition of the parameters of the simulation, we will present them with a advanced vocabulary:

\begin{verbatim}
   float frequencePaquets = 5.0;      // Nombre moyen de pq/s
   float tailleMoyenne    = 1000.0;   // Taille moyenne des pq
   float debit            = 10000.0;  // En bit par seconde
\end{verbatim}

   These values ​​are unrealistic, but cleverly chosen for
having the same results as the first tutorial.   

%........................................................................
%
%........................................................................
\subsubsection{Generation of packet size}

   This is the main difference from the first tutorial. We will
use a random number generator for different packets size:

\begin{verbatim}
   /* Création d'un générateur de taille (tailles non bornées) */
   sizeGen = randomGenerator_createUInt();
   randomGenerator_setDistributionExp(sizeGen, 1.0/tailleMoyenne);

   /* Affectation à la source */
   PDUSource_setPDUSizeGenerator(sourcePDU, sizeGen);
\end{verbatim}

  The random number generators are described in the section
\ref{section:rand-gen}. Place a probe of this size, in order to verify that it it's right:


\begin{verbatim}
   /* A sensor */
   szProbe = probe_createExhaustive();
   randomGenerator_setValueProbe(sizeGen, szProbe);
\end{verbatim}

   With this, we can draw more than a nice curve!

%........................................................................
%
%........................................................................
\subsubsection{The server}

   What we just did is useless if the server is
not considered. So we need to specify that it serves each client
in a time that dependends on size:
   

\begin{verbatim}
   /* Paramétrage du serveur */
   srvGen_setServiceTime(serveur, serviceTimeProp, 1.0/debit);
\end{verbatim}
   
   This means that the time of serving a client is proportional to
its size, having the coefficient as reverse flow.

