%========================================================================
%
%========================================================================
\section{A simple http request model}

 The implemented model represents a simplification of the method recommended in "Source traffic modeling of wireless applications".

The ON-OFF model suggested in the document is composed from an HTTP-ON phase and an HTTP-OFF phase.
\begin{description}
 \item[HTTP-ON] phase models the activity after accepting a web request.
 \item[HTTP-OFF] phase represebts the period when a user doesn't make any request. It's formed by the loading page time and the viewing time.
\end{description}

\subsection{HTTP model}

We have two sets of parameters:
\begin{itemize}
 \item the first set describes the session level
 \item the second set describes the composition of a web request
\end{itemize}

The recommended model contains the following random variables:

\begin{description}
 \item[I]: the time between two WWW sessions
 \item[X]: the number of requests per session
 \item[V]: the web request viewing time(HTTP-OFF time)
 \item[M]: the size of main object of a webpage
 \item[N]: the number of inline objects of a webpage
 \item[O]: the size of inline objects
 \item[R]: the size of \textit{GetRequests}

\end{description}
Thus being said, the random variable are distributed in the following way:

\begin{description}
 \item[I] - it is hard to establish a distribution, so it was used I = 1 second between two sessions;
 \item[X] - Lognormal distributed with a mean E = 25 and a standard deviation $\sigma$=100;
 \item[V] - Weibull distributed with a mean of 39.5 seconds( $\alpha$ = 0.5, $\beta$ = 4.44);
 \item[M] - Lognormal distributed ($\alpha$ = 1.31, $\beta$ = 1.41) - with a mean of 10kB;
 \item[N] - Gamma distributed ($\alpha$ = 0.24, $\beta$ = 23.42) - with a mean of 5.55 inline objects per webpage;
 \item[O] - Lognormal distributed ($\alpha$ = -0.75, $\beta$ = 2.36) - with a mean of 7.7kB;
 \item[R] - Lognormal distributed ($\alpha$ = 5.84, $\beta$ = 0.29) - with a mean of 360B;
\end{description}

\subsection{Implementation of the simulator}
At first, we describe a general simulator:
\begin{verbatim}

   // We create a general simulator
       motSim_create();          
       sink = PDUSink_create();
   // The recommended model uses for interarrival time a Weibull distribution
  dateGenWb = dateGenerator_createWeibull(alpha, beta);
  
  server = srvGen_create(sink, (processPDU_t)PDUSink_processPDU);
  srvGen_setServiceTime(server, serviceTimeProp, 1.0/debit); 

  filePDU = filePDU_create(server, (processPDU_t)srvGen_processPDU); 

\end{verbatim}

\subsubsection{The request}

The variable V described before represents the time when a user reads a page, the time between two consecutive requests.
 This being told, for the request part, I suggest the following code to set the simulator:

\begin{verbatim}

   /* The size of GetRequest packets is Lognormal distributed 
      with req_alpha = 5.84 and req_beta = 0.29
   */    
      getReqSzGen = randomGenerator_createDouble();
      randomGenerator_setDistributionLognormal(getReqSzGen, req_alpha, req_beta);
      
      getRequest = PDUSource_create(dateGenWb, filePDU, (processPDU_t)filePDU_processPDU); 
    /* We associate the PDUSource the random size generator getReqSzGen */
      PDUSource_setPDUSizeGenerator(reqPDU, getReqSzGen);

\end{verbatim}

 To start the simulator:
\begin{verbatim}
      PDUSource_start(getReqSzGen);
\end{verbatim}

\subsubsection{The Reply} 

The random variables for main object size, inline object average size and number use Lognormal and Gamma distributions.
 These are implemented in the source random-generator.c

To set the simualtor for a http reply:

\begin{verbatim}
      reqSizeGen = randomGenerator_createUInt();
      randomGenerator_setDistributionComposed(reqSizeGen, a, b, ain, bin, galpha, gbeta);
      
 
      reqPDU = PDUSource_create(dateGenWb, filePDU, (processPDU_t)filePDU_processPDU);  
  
      /* We associate the size of the request packet to the reqPDU */
      PDUSource_setPDUSizeGenerator(reqPDU, reqSizeGen);
\end{verbatim}

 To start the simulator:
\begin{verbatim}
PDUSource_start(reqPDU);
\end{verbatim}



