%========================================================================
%
%========================================================================
\section{Extension of the simulator}
\label{section:extension}


%------------------------------------------------------------------------
%
%------------------------------------------------------------------------
\subsection{Implementation of a scheduler}
\label{subsection:tut-ordo}

   For example, we will implement a {\em Round
Robin} scheduler. The source files and the {\tt Makefile} are in the directory {\tt tuto-prog-1}.

%........................................................................
%
%........................................................................
\subsubsection{The characteristics of the scheduler}

   Forget about the {\tt include} and a look at how the
scheduler is characterised:
  
\begin{verbatim}
#define SCHED_RR_NB_INPUT_MAX 8

struct rrSched_t {
   // The destination (basically a link)
   void         * destination;
   processPDU_t   destProcessPDU;

   // The sources (the input files)
   int        nbSources;
   void     * sources[SCHED_RR_NB_INPUT_MAX];
   getPDU_t   srcGetPDU[SCHED_RR_NB_INPUT_MAX];

   // The last source served by the tourniquet
   int lastServed;
};
\end{verbatim}

    We need to know the upstream entities, since we want to
treat them separately!

    Creating the function is quite simple:

\index{rrSched\_create}
\begin{verbatim}
struct rrSched_t * rrSched_create(void * destination,
				  processPDU_t destProcessPDU)
{
   struct rrSched_t * result = (struct rrSched_t * )sim_malloc(sizeof(struct rrSched_t));

   // Destination management
   result->destination = destination;
   result->destProcessPDU = destProcessPDU;

   // No source defined
   result->nbSources = 0;

   // It starts somewhere ...
   result->lastServed = 0;

   return result;
}
\end{verbatim}

    It manages the destination, like any entity that can provide
{\sc pdu}s should do, and we take care of its specificities.

    For example, a source will be assigned as follows:

\index{rrSched\_addSource}
\begin{verbatim}
void rrSched_addSource(struct rrSched_t * sched,
		       void * source,
		       getPDU_t getPDU)
{
   assert(sched->nbSources < SCHED_RR_NB_INPUT_MAX);

   sched->sources[sched->nbSources] = source;
   sched->srcGetPDU[sched->nbSources++] = getPDU;
}
\end{verbatim}

   Be careful, this is not very robust! But this is not the objective
here \ldots

%........................................................................
%
%........................................................................
\subsubsection{The function {\tt rrSched\_getPDU}}
 
   This function is invoked by the downstream entity for
obtaining a {\sc pdu}. It is in this function that we are going to
implement the scheduling algorithm.

    Here is the code

\index{rrSched\_getPDU}
\begin{verbatim}
struct PDU_t * rrSched_getPDU(void * s)
{
   struct rrSched_t * sched = (struct rrSched_t * )s;
   struct PDU_t * result = NULL;

   assert(sched->nbSources > 0);

   int next = sched->lastServed;

   // What is the next source to serve ?
   do {
      // We search for the next source who can give us something
      next = (next + 1)%sched->nbSources;
      result = sched->srcGetPDU[next](sched->sources[next]);
   } while ((result == NULL) && (next != sched->lastServed));

   if (result)
     sched->lastServed = next;
   return result;
}
\end{verbatim}

   
   In fact, there is not much to say! The scheduling algorithm is applied and it possibly provides a {\tt NULL} {\sc pdu} if there is nothing to schedule.

%........................................................................
%
%........................................................................
\subsubsection{The function {\tt rrSched\_processPDU}}

   Go to the {\tt rrSched\_processPDU} function
that is invoked by a source that has a {\sc pdu} and who whill be
passed to the scheduler.

    It should be treated if possible, but do not treat if
it can not be treat. Here is the beginning of the function

\index{rrSched\_processPDU}
\begin{verbatim}
int rrSched_processPDU(void *s,
		       getPDU_t getPDU,
		       void * source)
{
   int result;
   struct rrSched_t * sched = (struct rrSched_t *)s;

   printf_debug(DEBUG_SCHED, "in\n");
\end{verbatim}

    I used the debugging, for example. We simply make a {\em cast}
to respect the prototype functions.

    The first thing to do is to check availability of the
downstream entity:

\begin{verbatim}
   // La destination est-elle prete ?
   int destDispo = sched->destProcessPDU(sched->destination, NULL, NULL);
\end{verbatim}
  
   Now we will treat the case of test ({\tt NULL} parameters  to test our availability):

\begin{verbatim}
   // If it is a test available, I depend on the downstream
   if ((getPDU == NULL) || (source == NULL)) {
      result = destDispo;
\end{verbatim}

  If the downstream entity is ready, we told it to pick a {\sc pdu} (that one or another, it is the algorithm that will tell, but we do not see how it could
be another!). Otherwise, it drops, \ldots

\begin{verbatim}
   } else {
      if (destDispo) {
         // If the approval is available, he was told to pick a PDU and it
         // will start the scheduler
         result = sched->destProcessPDU(sched->destination, rrSched_getPDU, sched);
      } else {
         // It does not matter if the downstream (support a priori) is not ready
         result = 0;
      }
   }
\end{verbatim}

   On oublie pas de renvoyer le résultat !

\begin{verbatim}

   printf_debug(DEBUG_SCHED, "out\n");
   return result;
}
\end{verbatim}

%........................................................................
%
%........................................................................
\subsubsection{Use}

   Well, I let you read the source!

