%========================================================================
%
%========================================================================
\section{L'architecture générale}
\label{section:architecture}

   Le but de cette section est de décrire la logique de {\sc
ndes}.

%------------------------------------------------------------------------
%
%------------------------------------------------------------------------
\subsection{Type de simulation}

   {\sc Ndes} est une librairie de simulation de réseaux par événements
discrets. Non, ce n'est pas original, mais c'est tout de même
vachement important pour la suite. Tout le code de traitement d'un
événement que l'on va écrire sera exécuté en une durée éventuelllement
super longue, mais absolument pas comptabilisé dans le temps
simulé. Et pire ! Pendant le traitement d'un événement, comme le temps
simulé est figé, le système n'évolue pas (sauf au travers du code en
question). C'est du connu, d'accord, mais c'est lourd de conséquences
(musique stressante).

%------------------------------------------------------------------------
%
%------------------------------------------------------------------------
\subsection{Schéma général des modèles}

   L'idée fondamentale est que le réseau à simuler peut être
modélisé comme une suite d'entitées chaînées entre elles. Dans la
version la plus simple, strictement linéaire, la première de ces
entités va produire des messages, nommés {\sc pdu}s, qu'elle fera
suivre à l'entité suivante et ainsi de suite jusqu'à la dernière qui
pourra sonserver les messages ou les détruire.

   La file M/M/1 décrite dans le premier tutoriel est un parfait
exemple de ce modèle. Nous connaissons donc déjà quatre types de
noeuds : {\tt sourcePDU}, {\tt filePDU}, {\tt serv\_gen} et {\tt PDUSink}.

   La construction du réseau se fait en partant ``de la fin'' et en
remontant vers la source, comme on peut le voir dans le
tutoriel. Chaque fois que l'on crée une entité, on doit lui passer
comme paramètres des informations sur l'entité en aval. Elle n'a, en
revanche, aucun besoin de connaître la ou les entités amont. Oui, il
peut y en avoir plusieurs.

   Certaines entités peuvent également avoir plusieurs entités aval
(qui seront fournies par une fonction spécifique), mais une {\sc pdu}
donnée n'est passée qu'à une seule d'entre elles.

   La mauvaise nouvelle, c'est que que ce schéma va être à revoir tôt
ou tard ! Il n'est pas génial pour modéliser des choses plus
symétriques (une entité protocolaire qui émet et reçoit), et des
choses sont à revoir comme le terme de {\sc pdu} qui n'est pas du tout
approprié. Ce ne sont pas que des {\sc pdu} qui sont représentées par
cette chose !

%------------------------------------------------------------------------
%
%------------------------------------------------------------------------
\subsection{Fonctions d'échange des {\sc pdu}}

   Du fait de l'architecture générale du réseau, pour toute entité susceptible
de produire ou transférer des PDU doit exister une fonction de la forme

\begin{verbatim}
   struct PDU_t * getPDU(void * source);
\end{verbatim}

   Le paramètre est un pointeur vers des "données privées" permettant
d'identifier le noeud (typiquement un pointeur direct sur ce noeud).

   Le pointeur retourné est celui d'une PDU qui n'est plus prise en
compte par la source. Elle doit donc absolument être gérée (ou, au
moins, détruite) par l'utilisateur de cette fonction. En cas
d'indisponibilité de PDU, la valeur {\tt NULL} est retournée.

   Cette fonction et le pointeur associé doivent être fournis à
l'entité destinataire, s'il en existe une !

   Si le nom du module est {\tt toto}, la fonction sera nommée par exemple
{\tt toto\_getPDU().}\\

   Symétriquement, tout module susceptible de recevoir des PDU doit fournir une
fonction de la forme :


\begin{verbatim}
   int processPDU(void * rec,
                  getPDU_t getPDU,
                  void * source);
\end{verbatim}

   C'est cette fonction qu'invoquera une source pour lui notifier la
disponibilité d'une PDU. Cette fonction aura donc la responsabilité
d'aller récupérer la PDU (grâce à la fonction \lstinline!getPDU! et à la source
fournies) et de la traiter. La récupération et le traitement pourront
être remis à plus tard (en cas d'indisponibilité) mais au risque
d'avoir un pointeur \lstinline!NULL! retourné par
\lstinline!getPDU()!.

   La valeur de retour de cette fonction est nulle en cas d'échec, et
non nulle en cas de succès.

   Cette fonction peut être invoquée avec les deux derniers paramètres%l'avant-dernier paramètre%
{\tt NULL}. Dans ce cas, l'objectif est uniquement de déterminer si
l'entité {\tt rec} est disposée à traiter une nouvelle {\sc pdu}. Elle
doit donc renvoyer 0 ou 1 si elle est occupée ou libre, respectivement. 

   Cette propriété doit être implantée dans tous les cas et pourra
être utilisée par exemple par une entité amont qui peut conserver les
{\sc pdu}s (une file par exemple) afin d'éviter une perte dûe à une
entité aval qui ne pourrait pas recevoir une nouvelle {\sc pdu} (un
lien en cours de  transmission par exemple).

   Un exemple d'utilisation est donné dans le premier ''{\em Tutoriel
     du programmeur}'' en sous-section \ref{subsection:tut-ordo}.

   Si le nom du module est {\tt toto}, la fonction sera nommée par exemple
{\tt toto\_processPDU()}.

   Alors là tu vas me dire ``qu'est-ce que c'est que ce chantier
?!''. En fait, voilà comment c'est censé fonctionner (et ça a l'air
de marcher !). Imaginons deux entités {\tt A} et {\tt B} de notre
réseau, {\tt B} étant en aval de {\tt A} comme dans la figure
\ref{figure:reseauAB}.

%.......................................................................
%
%.......................................................................
\subsubsection{Transmission d'une {\sc pdu}}

   Si, lors du traitement d'un événement, l'entité {\tt A} doit faire
suivre une {\sc pdu} à {\tt B}, elle le fera en invoquant la fonction
\lstinline!processPDU! associée à {\tt B}. À partir de ce moment, la
{\sc pdu} est sous la responsabilité de {\tt B}. Deux cas sont
envisageables : 

\begin{itemize}
   \item Soit {\tt B} est prète à traiter la {\tt pdu} (c'est une file
     non pleine, ou un serveur inactif par exemple), alors il invoque
     immédiatement (dans le code de son \lstinline!processPDU!) la
     fonction \lstinline!getPDU! de {\tt A} et tout va bien.
   \item Soit {\tt B} n'est pas prète. Dans le code de son
     \lstinline!processPDU!, il n'y aura donc pas d'invocation du
     \lstinline!getPDU! de {\tt A}, mais il devra y avoir une action
     permettant de le faire plus tard\footnote{En fait, ce n'est pas
       tout à fait vrai ! Il faut être certain que {\tt B} ira
       récupérer cette {\tt pdu}, mais on peut s'arranger autrement
       comme on va le voir avec l'arrivée d'une {\sc pdu}} (dans le temps simulé !), par
     exemple positioner un booléen quelconque. Mais d'ici à ce que
     cette invocation arrive, peut-être {\tt A} aura détruit la {\sc
       pdu} (imagine que {\tt A} modélise une couche physique, elle ne
     va pas retenir une {\sc pdu}, enfin, sois raisonable !
\end{itemize}

%.......................................................................
%
%.......................................................................
\subsubsection{Arrivée d'une {\sc pdu}}

   D'un autre côté, si, lors du traitement d'un événement, l'entité
{\tt B} est prête à consommer une {\sc pdu}, elle le fera en invoquant
la fonction \lstinline!getPDU! associée à {\tt A}.

   Là aussi, deux cas sont envisagables

\begin{itemize}
   \item Soit {\tt A} dispose effectivement d'une {\sc pdu} à fournir
     à {\tt B}, et dans ce cas là tout roule !
   \item Soit {\tt A} n'a pas de {\sc pdu}. Dans ce cas, la fonction
     \lstinline!getPDU! va renvoyer un pointeur \lstinline!NULL! qu'il
     vaut donc être prèt à traiter.
\end{itemize}

   L'invocation de la fonction \lstinline!getPDU! de l'entité amont
peut donc être déclanchée par un événement qui n'a rien a voir avec
elle, c'est comme cela que l'on va aller chercher des {\sc pdu}s sans
y avoir été invité, d'où la note précédente, et d'où (entre autres) le
risque du pointeur \lstinline!NULL!.

%.......................................................................
%
%.......................................................................
\subsubsection{Chronologie des événements}

  Tout cela peut paraître un peu vicieux, et tu dois avoir du mal,
cher public, à voir quelle fonction utiliser quand ! Déjà,
rassure-toi, tout ça n'a besoin d'être maîtrisé que par celui qui veut
créer de nouveaux types d'entités pour le simulateur.

   D'autre part, c'est en fait très simple, il suffit de suivre la
logique des événements. Imaginons par exemple que tu veuilles
modéliser un ordonnanceur {\em round-robin}. On va supposer qu'il est
en aval d'un certain nombre de files, et en amont d'un serveur qui
modélise un lien de communication. Voir la figure
\ref{figure:exempleordo}.

% Figure

%
%
%
\paragraph{\lstinline!schedRRProcessPDU!}

   À quoi va ressembler le code de la fonction de traitement d'une
{\sc pdu} de cet ordonnanceur ? 

   Cette fonction sera invoquée lorsque l'une des files amont aura une
{\sc pdu} à fournir à l'ordonnanceur. Mais cette {\sc pdu} ne doit
être ordonnancée que si les deux conditions suivantes sont vérifiées :

\begin{itemize}
   \item le support (aval) est libre ;
   \item c'est au tour de cette file d'être servie ou les autres sont
     vides\footnote{Sinon il est non work conserving}.
\end{itemize}

   La fonction \lstinline!schedRRProcessPDU! va donc devoir tester ces
conditions et, si elles sont vraies, récupérer effectivement la {\sc
  pdu} (avec le \lstinline!getPDU! de la file) puis l'envoyer sur le
lien (avec le \lstinline!processPDU! du serveur). 

   Si les conditions ne sont pas vérifiées, alors elle laisse la {\sc
pdu} où elle est.

%
%
%
\paragraph{\lstinline!schedRRGetPDU!}

   Et maintenant à quoi ressemble la fonction d'obtention d'une {\sc
 pdu} de notre ordonnanceur ?

   Elle est invoquée par le support de communication (aval) lorsqu'il
est libre (un événement de fin de transmission de la précédente par
exemple).

   L'ordonnanceur ne dispose pas de {\sc pdu} lui même. Il doit donc
aller chercher dans les files amont la prochaine à servir et lui
prendre une {\sc pdu} par le biais de son \lstinline!getPDU!. S'il ne
trouve rien, il retourne un pointeur \lstinline!NULL! à l'entité aval
(le lien) qui ne fait donc rien.

   Dans cette situation le prochain événement sera l'arrivée d'une
{\sc pdu} dans une file par l'invocation de son \lstinline!processPDU!
qui invoquera celui de l'ordonnanceur, \ldots

%------------------------------------------------------------------------
%
%------------------------------------------------------------------------
\subsection{Plusieurs émetteurs vers le même destinataire}

   Comment faire lorsque plusieurs entités sont en amont d'une même
entitée aval ? Par exemple plusieurs sources qui envoient des clients
vers un unique serveur.

   En fait, ce n'est pas fondamentalement un problème en soit, dans la
mesure ou l'entité aval n'a en général pas à connaître l'entité
amont. Du coup, s'il y en a deux ou plus, elle s'en moque. Oui, sauf
que non ! Imagine une entité qui n'est pas toujours disposée à traiter
une {\sc pdu} entrante. En général, elle va stocker quelque part une
information qui lui permettra de venir chercher la {\sc pdu} quand
elle le voudra bien. Si ce phénomène se produit plusieurs fois avant
que l'entité soit disposée à enfin traiter les {\sc pdu}s en attente,
seule la dernière risque d'avoir été mémorisée !

   Du coup, on pourrait être tenté de faire une liste chaînée de ces 
événements, de sorte à n'en rater aucun. Malheureusement, cela met
nécessairement en place une politique {\sc fcfs}, alors qu'on peut
souhaiter implanter autre chose.

