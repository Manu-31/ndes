%========================================================================
%
%========================================================================
\section{Un multiplexeur/démultiplexeur}
\label{section:muxdemux}

   Un outil de multiplexage/démultiplexage élémentaire est implanté
dans {\tt src/muxdemux.c} définissant une entité de multiplexage et
une entité de démultiplexage. Le but est de fournir un cadre simple
mais propre permettant à l'utilisateur de multiplexer dans une même
séquence d'entités des flux différents et d'\^etre capable de les
différencier à l'arrivée.

   Attention, cet outil ne fournit aucune fonctionnalité de stockage,
si bien que si le premier élément de la séquences séparant le
multiplexeur du démultiplexeur n'est pas assez rapide, alors des {\sc
  pdu} peuvent être perdues.

%------------------------------------------------------------------------
%
%------------------------------------------------------------------------
\subsection{Principe général}

   Cet outil est composé de deux partie distinctes :

\begin{itemize}
   \item Une entité de type {\tt struct muxDemuxSender\_t} fait office
     de multiplexeur. Elle est associée à une destination unique et
     possède un nombre d'entrée quelconque.
   \item Une entité de type {\tt struct muxDemuxReceiver\_t } joue le
     rôle de démultiplexeur. Elle possède une seule entrée mais
     dessert un nombre quelconque de desinations.
\end{itemize}

   Une {\sc pdu} reçue par le multiplexeur sera encapsulée par ce
dernier  dans une nouvelle {\sc pdu} qui contiendra également les
informations permettant au démultiplexeur de faire son travail.

   Par défaut, la nouvelle {\sc pdu} ainsi créée a la même taille que
la {\sc pdu} d'origine.

%------------------------------------------------------------------------
%
%------------------------------------------------------------------------
\subsection{Multiplexeur : déclaration et initialisation}


%------------------------------------------------------------------------
%
%------------------------------------------------------------------------
\subsection{Démultiplexeur : déclaration et initialisation}

%------------------------------------------------------------------------
%
%------------------------------------------------------------------------
\subsection{Utilisation d'un filtre pour éviter le multiplexage}

   Il peut être intéressant de conditionner des traitements sur
l'identifiant de multiplexage sans démultiplexer les flux. Un filtre
peut être simplement construit pour cela. Les filtres sont décrits
dans la section \ref{section:filtres}.

\index{muxDemuxSender\_createFilterFromSAP}
\begin{verbatim}
struct PDUFilter_t * muxDemuxSender_createFilterFromSAP(struct muxDemuxSenderSAP_t * sap);
\end{verbatim}

   Un exemple est disponible dans le fichier de test {\tt tests/rr-mux.c}.
