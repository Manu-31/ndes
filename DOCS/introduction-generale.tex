%========================================================================
%
%========================================================================
\section{Introduction générale}

   {\em Network Discrete Event Simulator} (ou {\sc ndes}) est une
librairie permettant de construire des simulateurs réseaux à
événements discret. Cette librairie peut être utilisée à différentes
fins. Nous classerons les individus en deux catégories en fonction de
leurs objectifs.

\begin{description}
   \item[Un utilisateur] est une personne souhaitant écrire un
     programme de simulation fondé sur cette librairie. Il n'a besoin
     que de la version compilée de cette librairie. Il ne lui est pas
     nécessaire d'en connaître les détails de fonctionnement.
   \item[Un programmeur] est une personne visant à améliorer, enrichir
     ou déboguer cette librairie. Il a donc besoin de manipuler le
     code source et, pour cela, d'en comprendre le fonctionnement.
\end{description}

   La distinction est importante pour la suite du
document.

   L'utilisateur pourra commencer par lire la section
\ref{section:user-tuto} qui présente l'utilisation du simulateur sous
la forme d'un tutoriel (dont le code est fourni avec les source de
{\sc ndes}). Il pourra ensuite consulter les sections présentant les
outils du simulateur (les sondes, les générateurs, \ldots) et la
librairie des modèles.

   Le programmeur, quant à lui, devra lire la section dédiée à
l'extension du simulateur, là aussi présentée sous la forme d'un
tutoriel, puis consulter les sections décrivant les arcanes de {\sc
  ndes} telles que le moteur de simulation.

   Comme souvent dans ce genre de projet, malheureusement, il est
illusoire d'espérer une documentation à jour. L'objectif est avant
tout de développer l'outil afin d'en améliorer l'usage ciblé que de
maintenir une documentation. Le lecteur est donc invité à consulter
les sources suivantes d'information, dans l'ordre inverse de fraîcheur
:

\begin{itemize}
   \item ce document ;
   \item la documentation extraite par doxygen des sources et présente
     dans le répertoire {\tt DOCS} ;
   \item le code source.
\end{itemize}

