%========================================================================
%
%========================================================================
\section{Les {\sc pdu}s}
\label{section:pdu}

   La {\sc pdu} est l'objet de base échangé entre les entités du
simulateur. Si on voit considère deux entités homologues qui
dialoguent, il est effectivement légitime d'utiliser ce
terme. Cependant, dans une simulation réseau faite avec {\sc ndes}, la
notion d'entité est bien plus vaste que cela si bien que ce nom est
finalement mal choisi ! N'empèche que pour le moment, je vais le
conserver !

   Les {\sc pdu}s sont définies dans le fichier {\tt pdu.h} et
implantées dans le fichier {\tt pdu.c}. Une {\sc pdu} est une
structure extrèmement légère, dotée de quelques caractéristiques de
base : un identifiant, une taille, une date de création (dans le temps
simulé), et des données privées (un pointeur).

%------------------------------------------------------------------------
%
%------------------------------------------------------------------------
\subsection{Création/destruction}

   On la crée avec la fonction

\index{PDU\_create}
\begin{verbatim}
struct PDU_t * PDU_create(int size, void * private);
\end{verbatim}

   Le paramètre {\tt size} est évidemment la taille en octets, et {\tt
private} peut être un pointeur vers des données privées associées à
cette {\sc pdu} (il peut être {\tt NULL}, personnellement, je m'en
fiche).

%------------------------------------------------------------------------
%
%------------------------------------------------------------------------
\subsection{Les caractéristiques de base}

   La taille d'une {\sc pdu} est donnée par

\index{PDU\_size}
\begin{verbatim}
int PDU_size(struct PDU_t * PDU);
\end{verbatim}

%------------------------------------------------------------------------
%
%------------------------------------------------------------------------
\subsection{Les données privées}

