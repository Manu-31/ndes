\documentclass{book}
\usepackage{xr}
\externaldocument{random-generator-en.tex}
\usepackage{amsmath}
\usepackage[utf8x]{inputenc}
\usepackage[french]{babel}
\usepackage{graphicx}
\usepackage{srcltx}
\usepackage{listings}
\usepackage[pdftex]{thumbpdf}
\usepackage[pdftex,colorlinks]{hyperref}
\usepackage{index}
\usepackage{hyperref}


\title{Network Discrete Event Simulator}

\author{Emmanuel Chaput}

\makeindex

\begin{document}
\maketitle

%\begin{abstract}
   
    {\em Network Discrete Event Simulator} (or {\sc ndes}) is one
library which permits the creation of network discrete event simulators. This  document is a user guide and a programmer guide for this utility.

   {\sc Ndes} was basically designed to simulate a
specific scenario (scheduling of packets on a DVB-S2 link). 
But it also includes various portions of code with whom
the author had other goals. Its use in a different context is however very difficult, especially because of the absence of a substantial library of models.

%\end{abstract}

\newpage
\tableofcontents
\newpage

%========================================================================
%
%========================================================================
\chapter{User's manual}

%========================================================================
%
%========================================================================
\section{General Introduction}

   {\em Network Discrete Event Simulator} (or {\sc ndes}) is a library
that permits the creation of discrete event simulators. This library can be used for different purposes. We classify individuals into two categories according to
their goals.

\begin{description}
   \item[An user] is a person who wishes to write a
      simulation program based on this library. He only needs
      the compiled version of this library. It is not
      necessary for him to know the details of the operation.

   \item[A programmer] is a person who wants to improve and enrich
      or debug the library. So he needs to handle the
      source code, and for that, to understand the operation.
\end{description}

   The difference between these two is important for this document.

   The user can start with reading the section
\ref{section:user-tuto} who presents the use of a simulator in a tutorial 
(whose code is included with the {\sc ndes} source). He can also consult the sections that presents the tools of the simulator (the sensors, the generators, \ldots) and the library of models.

   The programmer, meanwhile, should read the dedicated extended section of the simulator, as shown here in the form of a tutorial, then he should look at the sections describing the complications of {\sc ndes}, by example, the simulation engine.

   
   As it often happens in this kind of project, unfortunately, it is
unrealistic to expect a documentation update. The primary objective is
developing any tool in order to improve the targeted use of that
maintain documentation. The reader is invited to consult
the following sources of information, in reverse order of chronology:


\begin{itemize}
   \item this document ;
   \item extracted documentation of sources, presented
      in the directory {\tt DOCS} ;
   \item the source code.
\end{itemize}


%========================================================================
%
%========================================================================
\section{Tutorial}
\label{section:user-tuto}

%------------------------------------------------------------------------
\subsection{Introduction}

   Congratulations! You are ready to enter the fabulous world of
{\sc ndes}, \ldots
 
    It's basically an artfully embellished library with numerous bugs and ready to help you (or not) to your own network simulator!

    The implementation of a simulator is done by writing a
C program using this library.


%------------------------------------------------------------------------
\subsection{Installing}

\begin{verbatim}
 $ git clone https://github.com/Manu-31/ndes.git
 $ cd ndes
 $ make
 $ make install
\end{verbatim}

   Do not worry, it does not install anything yet! This does
as copying a library (static) in a dedicated directory.

    We can also compile some test programs:

\begin{verbatim}
 $ make tests-bin
\end{verbatim}
%$
    And why not, to run it:

\begin{verbatim}
 $ make tests
\end{verbatim}
%$

   Obviously, the {\tt OK} messages don't prove us anything! Error messages tell us that there are problems \ldots {} as we were expected.

%------------------------------------------------------------------------
\subsection{My first simulation : the file M/M/1 !}

   The source file (and its Makefile, because we respect our client) is found in the directory {\tt example/tutorial-1}.

   In {\sc ndes}, the system will be modeled by a source, followed by a queue, downstream of which it exists a monitoring server followed by a sink. I leave you to draw a picture and I will publish here the prettiest one!

%........................................................................
\subsubsection{Creating a simulator}

   Before any action, we create the simulator in the following way:

\begin{verbatim}
#include <motsim.h>

...

   /* Create a simulator */
   motSim_create();
\end{verbatim}

%........................................................................
\subsubsection{Creating a sink}

   A sink is an object that receives messages without complaining and
which destroys them instantly. It is very simple to create a sink:

\begin{verbatim}
#include <pdu-sink.h>

...

   struct PDUSink_t       * sink; // Déclaration d'un puits

   ...

   /* Create a sink */
   sink = PDUSink_create();
\end{verbatim}

%........................................................................
%
%........................................................................
\subsubsection{Creating a server}
   
   As we do not want to do anything intelligent with our server, we will use a generic server like this:


\begin{verbatim}
#include <srv-gen.h>

...

   struct srvGen_t        * serveur; // Déclaration d'un serveur générique

...

   /* Create a server */
   serveur = srvGen_create(sink, (processPDU_t)PDUSink_processPDU);
\end{verbatim}

   The creation of the server is somewhat more complicated than that of
sinks. The reason is that the server firstly needs to know who to send
clients ({\sc pdu}s for {\sc ndes}). It must therefore say what function they apply (here {\tt PDUSink\_processPDU} a specific function sink) and how an
entity applies this function (our single sink, here). To know the function, it must be sought in the description of the target entity. In our example, the function {\tt PDUSink\_processPDU} is described in the section dedicated to the sinks.

    More details about these functions are found in the section
\ref{section:architecture} which describes the overall architecture.

    As we want a M/M/1 queue, we need to tell our server that its processing time is an exponential parameter {\tt mu}.

  
\begin{verbatim}
   float   mu = 10.0; // parameter of the server

   ...

   /* Set server's parameters */
   srvGen_setServiceTime(serveur, serviceTimeExp, mu);
\end{verbatim}

   Generic servers are described more specifically in the
section \ref{section:srv_gen}.

%........................................................................
%
%........................................................................
\subsubsection{Creating a file}
 
   A file allows to store the {\sc pdu}s in transit. We can create a file 
in the following way: 
  

\begin{verbatim}
   struct filePDU_t       * filePDU; // Declaration of a file pdu

    ...

   /* Create a file pdu */
   filePDU = filePDU_create(serveur, (processPDU_t)srvGen_processPDU);
\end{verbatim}

   Without further add, the files are not bounded. They
are precisely described in section \ref{section:file}. The
two parameters of the function to create the file have the same role as those of the function to create the server.
 
%........................................................................
%
%........................................................................
\subsubsection{Creating a source}

   We will use an object of {\sc ndes} whose role is to produce {\sc pdu}s. But before that, we need to create another object that indicates the time when it produces them: this is a "time generator". Time generators are described in section \ref{section:date_generator}.

    We want a poisonn source, in conclusion an exponential generator:

\begin{verbatim}
#include <date-generator.h>

...

   struct dateGenerator_t * dateGenExp; // Un générateur de dates
   float   lambda = 5.0 ; // Intensité du processus d'arrivée

   ...

   /* Create a data generator */
   dateGenExp = dateGenerator_createExp(lambda);
\end{verbatim}

   And now we can create our source:

\begin{verbatim}
#include <pdu-source.h>

...

   struct PDUSource_t     * sourcePDU;  // Une source

   ...

   sourcePDU = PDUSource_create(dateGenExp, 
				filePDU,
				(processPDU_t)filePDU_processPDU);
\end{verbatim}
  
   The first parameter is the object which allows him to determine
the date of sending. The following two are similar to those passed
when creating the file and the server.
   
%........................................................................
%
%........................................................................
\subsubsection{The implementation of probes}

   
   Yes, I know, the title is a little scary, but it will be simple. 
When you want to run a simulator, it is generaly hoped to get results.
   In {\sc ndes}, they will be collected by a specific tool, the probe(the sensor).

   Different types of sensors are described in section \ref{section:sondes}.
   Here we only use exhaustive probes. For each object described above, the list of available probes is provided.   

   We declare and initialize them like this:   

\begin{verbatim}
#include <probe.h>

...

   struct probe_t         * sejProbe, * iaProbe, * srvProbe; // Les sondes

...

   /* A sensor for inter-arrival time */
   iaProbe = probe_createExhaustive();
   dateGenerator_addInterArrivalProbe(dateGenExp, iaProbe);

   /* A sensor for journey time */
   sejProbe = probe_createExhaustive();
   filePDU_addSejournProbe(filePDU, sejProbe);

   /* A sensor for service time */
   srvProbe = probe_createExhaustive();
   srvGen_addServiceProbe(serveur, srvProbe);
\end{verbatim}

%........................................................................
%
%........................................................................
\subsubsection{Launch of the simulation}
   
   
   That's it, we're finally ready to start the
simulation. We must activate the desired entities at the appropriate
time. Here there is only a single source to activate, and we wish
to do it from the beginning of the simulation:


\begin{verbatim}
   /* We activate de source */
   PDUSource_start(sourcePDU);
\end{verbatim}

   We are now launching the simulation. We'll make it last
100 seconds:

\begin{verbatim}
   /* The simulation lasts 100 s = 100 000ms */
   motSim_runUntil(100000.0);
\end{verbatim}

   We can then display some internal settings of the simulator:

\begin{verbatim}
   motSim_printStatus();
\end{verbatim}

   And that's it!

%........................................................................
%
%........................................................................
\subsubsection{Showing the results}
  
   Now that our simulation is complete, we certainly
want to see the result. We use for this the functions
provided by the probes, for example:  


\begin{verbatim}
   /* Print scalar results */
   printf("%d packets in a file\n",
	  filePDU_length(filePDU));
   printf("Average time of journey = %f\n",
	  probe_mean(sejProbe));
   printf("Average inter-arival   = %f (1/lambda = %f)\n",
	  probe_mean(iaProbe), 1.0/lambda);
   printf("Average time of service = %f (1/mu     = %f)\n",
	  probe_mean(srvProbe), 1.0/mu);
\end{verbatim}

   Great, isn't it? No! There's something more \ldots

%........................................................................
%
%........................................................................
\subsubsection{Drawing the curves}

   
   For better results, we use(from the simulator) a display {\em gnuplot}. We have here at least two interesting plot curves, so we will write a
function for that:

\begin{verbatim}

/*
 * The results will be displayed in a graphbar
 * with nbBar bars with the name "name"
 */
void tracer(struct probe_t * pr, char * name, int nbBar)
{
   struct probe_t   * gb;
   struct gnuplot_t * gp;

   /* The source for the graphbar */
   gb = probe_createGraphBar(probe_min(pr), probe_max(pr), nbBar);

   /* We convert the sensor into a parameter for graphbar */
   probe_exhaustiveToGraphBar(pr, gb);

   /* We name it */
   probe_setName(gb, name);

   /* We initialise the gnuplot */
   gp = gnuplot_create();

   /* We set the range of the graph */
   gnuplot_setXRange(gp, probe_min(gb), probe_max(gb)/2.0);

   /* We display it */
   gnuplot_displayProbe(gp, WITH_BOXES, gb);
}
\end{verbatim}

   The use of {\em gnuplot} is described in the section
\ref{section:gnuplot}.

   Please do not forget to add a short break at the end of our main program, if not it stops and kills the child processes, and therefore the display {\rm gnuplot} dissapears.

%........................................................................
%
%........................................................................
\subsubsection{The use of our first simulator}

   It only remains for us to compile our program and to launch it. The {\tt examples/tutorial-1} also contains a makefile, but firstly, go and take a
look at the includes and the library. Use the {\tt Makefile} like this:

\begin{verbatim}
 $ cd examples/tutorial-1
 $ make
 $ ./mm1
[MOTSI] Date = 99999.846555
[MOTSI] Events : 998243 created (3 m + 998240 r)/998242 freed
[MOTSI] Simulated events : 998243 in, 998242 out, 1 pr.
[MOTSI] PDU : 998242 created (27 m + 998215 r)/998242 freed
[MOTSI] Total malloc'ed memory : 25169976 bytes
[MOTSI] Realtime duration : 1 sec
0 paquets restant dans  la file
Temps moyen de sejour dans la file = 0.061008
Interarive moyenne     = 0.200352 (1/lambda = 0.200000)
Temps de service moyen = 0.100176 (1/mu     = 0.100000)
*** ^C pour finir ;-)
 $
\end{verbatim}

%------------------------------------------------------------------------
%
%------------------------------------------------------------------------
\subsection{The second simulation : another M/M/1 file !}

   The first example is nice, but it would be better
if the processing time depends on the size of the customer (as
packets whose transmission time depends on the size!)

   This is what we will do in this second tutorial. The files are located in the directory {\tt example/tutorial-2}.

   I'll just comment the differences of this tutorial with the first one.

   A small cosmetic change is made in the definition of the parameters of the simulation, we will present them with a advanced vocabulary:

\begin{verbatim}
   float frequencePaquets = 5.0;      // Nombre moyen de pq/s
   float tailleMoyenne    = 1000.0;   // Taille moyenne des pq
   float debit            = 10000.0;  // En bit par seconde
\end{verbatim}

   These values ​​are unrealistic, but cleverly chosen for
having the same results as the first tutorial.   

%........................................................................
%
%........................................................................
\subsubsection{Generation of packet size}

   This is the main difference from the first tutorial. We will
use a random number generator for different packets size:

\begin{verbatim}
   /* Création d'un générateur de taille (tailles non bornées) */
   sizeGen = randomGenerator_createUInt();
   randomGenerator_setDistributionExp(sizeGen, 1.0/tailleMoyenne);

   /* Affectation à la source */
   PDUSource_setPDUSizeGenerator(sourcePDU, sizeGen);
\end{verbatim}

  The random number generators are described in the section
\ref{section:rand-gen}. Place a probe of this size, in order to verify that it it's right:


\begin{verbatim}
   /* A sensor */
   szProbe = probe_createExhaustive();
   randomGenerator_setValueProbe(sizeGen, szProbe);
\end{verbatim}

   With this, we can draw more than a nice curve!

%........................................................................
%
%........................................................................
\subsubsection{The server}

   What we just did is useless if the server is
not considered. So we need to specify that it serves each client
in a time that dependends on size:
   

\begin{verbatim}
   /* Paramétrage du serveur */
   srvGen_setServiceTime(serveur, serviceTimeProp, 1.0/debit);
\end{verbatim}
   
   This means that the time of serving a client is proportional to
its size, having the coefficient as reverse flow.


%\input{exemples-utilisateur-en}

%========================================================================
%
%========================================================================
\chapter{Programmer's manual}

%========================================================================
%
%========================================================================
\section{General architecture}
\label{section:architecture}

   
The purpose of this section is to describe the logic of {\sc
ndes}.

%------------------------------------------------------------------------
%
%------------------------------------------------------------------------
\subsection{Type of the simulation}

   {\sc Ndes} is a library of a discrete event network
simulator. No, this is not original, but it is still
really important for the future. All the code for processing a
event will be executed into a long time, but absolutely not recognized in simulated time. And worse! During the processing of an event, such as time
simulated is fixed, the system does not change (except through the code in
question). 

%------------------------------------------------------------------------
%
%------------------------------------------------------------------------
\subsection{General schema of the models}

   The basic idea is that the simulated network can be
modeled as a sequence of entitées chained together. In the
strictly linear simplest version, the first of these
entities will produce messages, called {\sc pdu}s, it will
follow the following entity and so on until the last one
may conserve the messages or destroy them.

   The M/M/1 queue described in the first tutorial is a perfect
example of this model. We already know the four types of
nodes: {\tt sourcePDU}, {\tt filePDU}, {\tt serv\_gen} and {\tt PDUSink}.

   The construction of the network is starting "later" and
based on the source, as it can be seen in the
tutorial. Each time you create an entity, you must pass
parameters as information about the downstream entity. 
However, we don't need to know anything about these entities. Yes, it
there can be several.

   Some entities may also have multiple downstream entities
(to be provided by a specific function), but the {\sc pdu}
data is passed to only one of them.

   
    The bad news is that this scheme will be reviewed soon
or later! It is not good to model things more
symmetric (a protocol entity that sends and receives), and
things to be reviewed as the term {\sc pdu} is not at all
appropriate. It is just {\sc pdu} which is represented by
this thing!

%------------------------------------------------------------------------
%
%------------------------------------------------------------------------
\subsection{The exchange {\sc pdu} function}

   Because of the general network architecture for any entity capable
generating or transferring PDU must be a function of the form

\begin{verbatim}
   struct PDU_t * getPDU(void * source);
\end{verbatim}

   The parameter is a pointer to the "private data" for
identify the node(typically a direct pointer to the node).

   The returned pointer is a PDU which is used by the source. 
It must absolutely be managed (or, less, destroyed) by the user of this function. 
If PDU is unavailable, the {\tt NULL} value is returned.

   This function and the associated pointer must be provided to
the recipient entity, if any!

    If the module name is {\tt foo}, the function will be named for example
{\tt foo\_getPDU().}\\

    Symmetrically, any module that may receive PDU must provide
function of the form:

\begin{verbatim}
   int processPDU(void * rec,
                  getPDU\_t getPDU,
                  void * source);
\end{verbatim}

   
    It is this function who invokes a source to notify the
availability of a PDU. This function will have the responsibility
go get the PDU (through the function \lstinline!getPDU! and the source included) 
and to treat it. The recovery and treating may
be postponed (if unavailable) but there is a risk to 
have a \lstinline!NULL! pointer returned by
\lstinline!getPDU()!.

   The value returned by this function is zero in case of failure, and
non-zero if successful.

    This function can be invoked with the last two parameters
{\tt NULL}. In this case, the only purpose is to determine if {\tt rec}
entity is ready to process a new {\sc pdu}. It
must return 0 if it is occupied or 1 if it is free.

   This property must be implemented in all cases and may
for example be used by an upstream entity that can keep
{\sc pdu}s (a file, for example) to avoid loss due to
a downstream entity that could not receive a new {\sc pdu} (a
link during transmission, for example).

   An example of utilisation is given in the first ''{\em Programmer's tutorial}'' in sub-section \ref{subsection:tut-ordo}.

   If the module name is {\tt foo}, the function will be named for example
{\tt foo\_processPDU()}.

    An then you tell me ``what is this site?!''. In fact, that's how it's supposed to work (and it looks like it's working!). Imagine two entities {\tt A} and {\tt B} of our
network, {\tt B} being the downstream of {\tt A} as in Figure \ref{figure: reseauAB}.

%.......................................................................
%
%.......................................................................
\subsubsection{The transmision of a {\sc pdu}}

   
   If, while processing an event, the entity {\tt A} must
follow a {\sc pdu} to {\tt B}, it will do by invoking the function
\lstinline!processPDU! associated with {\tt B}. From that moment, the
{\sc} pdu is the responsibility of {\tt B}. Two cases are
possible:

\begin{itemize}
   \item Either {\tt B} is ready to process the {\tt pdu} (this is a file
      not full, or an inactive server), then it invokes
      immediately (within its code \lstinline!processPDU!) the
      function \lstinline!getPDU! of {\tt A} and all is good.
   \item Either {\tt B} is not ready. In the code of the
      \lstinline!processPDU!, there will be no invocation of
      \lstinline!getPDU! of {\tt A}, but there must be an action
      to do it later \footnote{In fact, it is not
      quite true! We must be sure that {\tt B} will
      recover the {\tt pdu}, but we can make other arrangements
      as we will see with the arrival of a {\sc pdu}} (in the simulated time!). 
      But for that invocation to happen, perhaps {\tt A} has destroyed the {\sc pdu} 
      (guess {\tt A} models a physical layer, it
      will not hold a {\sc pdu}. Finally, it's reasonable!
\end{itemize}

%.......................................................................
%
%.......................................................................
\subsubsection{The arrival of a {\sc pdu}}

   On the other hand, if, during the processing of an event, the entity
{\tt B} is ready to consume a {\sc pdu}, it will by invoking
the function \lstinline!getPDU! associated with {\tt A}.

    Again, two cases have been recognised:

\begin{itemize}
   \item Either {\tt A} has actually a {\sc pdu} to provide
      to {\tt B}, and in this case everything goes well!
   \item Either {\tt A} doesn't have {\sc pdu}s. In that case, the function
     \lstinline!getPDU! will transmit a \lstinline!NULL! pointer who
     will be ready to treat.
\end{itemize}

   The invocation of the function \lstinline!GetPDU! can be triggered by an event that has nothing to do with it, this is how we will get the {\sc pdu}s without
having been requested, where the previous note, and (among other things) it 
risks a \lstinline!NULL! pointer.

%.......................................................................
%
%.......................................................................
\subsubsection{The chronology of the events}

   All of this may seem a bit vicious, and you may be unclear to know what 
function to use when! But do not worry, all that has to be mastered by anyone 
who wants create new types of entities in the simulator.

    On the other hand, it is actually very simple, just follow the
logic of events. For example, imagine that you want to
model a {\em round-robin} scheduler. We will assume that it is
downstream of a number of files, and upstream of a server that
models a communication link. See Figure
\ref {figure: exempleordo}.    

% Figure

%
%
%
\paragraph{\lstinline!schedRRProcessPDU!}

   How it will look like the code for the treatment of a {\sc pdu} in this scheduler?

    This function is invoked when one of the upstream queues will have a
{\sc pdu} to provide to the scheduler. But the {\sc pdu} must
be scheduled only if the following two conditions are satisfied:
\begin{itemize}
   \item the support (downstream) is free;
   \item it is the turn of the queue to be served or the other is
      empty \footnote{If not it will not work conserving}.
\end{itemize}

   The function \lstinline!SchedRRProcessPDU! will therefore have to test these
conditions and, if they are true, actually retrieve the {\sc pdu} (with \lstinline!getPDU! call) and then to send the
link (with \lstinline!processPDU! of the server).

    If the conditions are not satisfied, it leaves the {\sc pdu} where it is.

%
%
%
\paragraph{\lstinline!schedRRGetPDU!}

   And now to whom it looks like this function to obtain a {\sc pdu} our scheduler?

   It is invoked by the communication support(downstream) when
it is free (an event of the end of the transmission from the preceding
example).

   The scheduler does not have {\sc pdu}s itself. It should
pick up in queues upstream next to him and serve
a {\sc pdu} through his \lstinline!getPDU!. If
it finds anything, it returns a \lstinline!NULL! pointer to downstream entity
(link) which therefore does nothing.

   In this situation, the next event is the arrival of a
{\sc pdu} in a queue by invoking the \lstinline!processPDU! function
that invoke the scheduler, \ldots

%------------------------------------------------------------------------
%
%------------------------------------------------------------------------
\subsection{Several transmitters to the same recipient}

   How about when several entities are upstream of the same
entitée downstream? For example, several sources that send customers
to a single server.

   In fact, this is not a problem, the downstream entity does not generally understand the entity upstream . So, if there are two or more, it doesn't care. 
Yes, except no! Imagine an entity that is not always prepared to deal with an
incoming {\sc pdu}. In general, it will store somewhere a
information that will allow him to pick the {\sc pdu} when it wants. 
If this happens several times before the entity is willing to finally treat the waiting{\sc pdu}s, only the last may have been stored !

   So, we might be tempted to make a linked list of
events, so as not to miss any . Unfortunately , this puts
necessarily a policy {\sc FCFS }, then we want to implement something else.


%========================================================================
%
%========================================================================
\section{Extension of the simulator}
\label{section:extension}


%------------------------------------------------------------------------
%
%------------------------------------------------------------------------
\subsection{Implementation of a scheduler}
\label{subsection:tut-ordo}

   For example, we will implement a {\em Round
Robin} scheduler. The source files and the {\tt Makefile} are in the directory {\tt tuto-prog-1}.

%........................................................................
%
%........................................................................
\subsubsection{The characteristics of the scheduler}

   Forget about the {\tt include} and a look at how the
scheduler is characterised:
  
\begin{verbatim}
#define SCHED_RR_NB_INPUT_MAX 8

struct rrSched_t {
   // The destination (basically a link)
   void         * destination;
   processPDU_t   destProcessPDU;

   // The sources (the input files)
   int        nbSources;
   void     * sources[SCHED_RR_NB_INPUT_MAX];
   getPDU_t   srcGetPDU[SCHED_RR_NB_INPUT_MAX];

   // The last source served by the tourniquet
   int lastServed;
};
\end{verbatim}

    We need to know the upstream entities, since we want to
treat them separately!

    Creating the function is quite simple:

\index{rrSched\_create}
\begin{verbatim}
struct rrSched_t * rrSched_create(void * destination,
				  processPDU_t destProcessPDU)
{
   struct rrSched_t * result = (struct rrSched_t * )sim_malloc(sizeof(struct rrSched_t));

   // Destination management
   result->destination = destination;
   result->destProcessPDU = destProcessPDU;

   // No source defined
   result->nbSources = 0;

   // It starts somewhere ...
   result->lastServed = 0;

   return result;
}
\end{verbatim}

    It manages the destination, like any entity that can provide
{\sc pdu}s should do, and we take care of its specificities.

    For example, a source will be assigned as follows:

\index{rrSched\_addSource}
\begin{verbatim}
void rrSched_addSource(struct rrSched_t * sched,
		       void * source,
		       getPDU_t getPDU)
{
   assert(sched->nbSources < SCHED_RR_NB_INPUT_MAX);

   sched->sources[sched->nbSources] = source;
   sched->srcGetPDU[sched->nbSources++] = getPDU;
}
\end{verbatim}

   Be careful, this is not very robust! But this is not the objective
here \ldots

%........................................................................
%
%........................................................................
\subsubsection{The function {\tt rrSched\_getPDU}}
 
   This function is invoked by the downstream entity for
obtaining a {\sc pdu}. It is in this function that we are going to
implement the scheduling algorithm.

    Here is the code

\index{rrSched\_getPDU}
\begin{verbatim}
struct PDU_t * rrSched_getPDU(void * s)
{
   struct rrSched_t * sched = (struct rrSched_t * )s;
   struct PDU_t * result = NULL;

   assert(sched->nbSources > 0);

   int next = sched->lastServed;

   // What is the next source to serve ?
   do {
      // We search for the next source who can give us something
      next = (next + 1)%sched->nbSources;
      result = sched->srcGetPDU[next](sched->sources[next]);
   } while ((result == NULL) && (next != sched->lastServed));

   if (result)
     sched->lastServed = next;
   return result;
}
\end{verbatim}

   
   In fact, there is not much to say! The scheduling algorithm is applied and it possibly provides a {\tt NULL} {\sc pdu} if there is nothing to schedule.

%........................................................................
%
%........................................................................
\subsubsection{The function {\tt rrSched\_processPDU}}

   Go to the {\tt rrSched\_processPDU} function
that is invoked by a source that has a {\sc pdu} and who whill be
passed to the scheduler.

    It should be treated if possible, but do not treat if
it can not be treat. Here is the beginning of the function

\index{rrSched\_processPDU}
\begin{verbatim}
int rrSched_processPDU(void *s,
		       getPDU_t getPDU,
		       void * source)
{
   int result;
   struct rrSched_t * sched = (struct rrSched_t *)s;

   printf_debug(DEBUG_SCHED, "in\n");
\end{verbatim}

    I used the debugging, for example. We simply make a {\em cast}
to respect the prototype functions.

    The first thing to do is to check availability of the
downstream entity:

\begin{verbatim}
   // La destination est-elle prete ?
   int destDispo = sched->destProcessPDU(sched->destination, NULL, NULL);
\end{verbatim}
  
   Now we will treat the case of test ({\tt NULL} parameters  to test our availability):

\begin{verbatim}
   // If it is a test available, I depend on the downstream
   if ((getPDU == NULL) || (source == NULL)) {
      result = destDispo;
\end{verbatim}

  If the downstream entity is ready, we told it to pick a {\sc pdu} (that one or another, it is the algorithm that will tell, but we do not see how it could
be another!). Otherwise, it drops, \ldots

\begin{verbatim}
   } else {
      if (destDispo) {
         // If the approval is available, he was told to pick a PDU and it
         // will start the scheduler
         result = sched->destProcessPDU(sched->destination, rrSched_getPDU, sched);
      } else {
         // It does not matter if the downstream (support a priori) is not ready
         result = 0;
      }
   }
\end{verbatim}

   On oublie pas de renvoyer le résultat !

\begin{verbatim}

   printf_debug(DEBUG_SCHED, "out\n");
   return result;
}
\end{verbatim}

%........................................................................
%
%........................................................................
\subsubsection{Use}

   Well, I let you read the source!


%\input{motsim-en}
%\input{ndes-object-en}
%\input{log-en}

\chapter{Traffic models}
%========================================================================
%
%========================================================================
\section{A simple http model}

 The implemented model represents a simplification of the method recommended in "Source traffic modeling of wireless applications".

The ON-OFF model suggested in the document is composed from an HTTP-ON phase and an HTTP-OFF phase.
\begin{description}
 \item[HTTP-ON] phase models the activity after accepting a web request.
 \item[HTTP-OFF] phase represebts the period when a user doesn't make any request. It's formed by the loading page time and the viewing time.
\end{description}

\subsection{HTTP model}

We have two sets of parameters:
\begin{itemize}
 \item the first set describes the session level
 \item the second set describes the composition of a web request
\end{itemize}

The recommended model contains the following random variables:

\begin{description}
 \item[I]: the time between two WWW sessions
 \item[X]: the number of requests per session
 \item[V]: the web request viewing time(HTTP-OFF time)
 \item[M]: the size of main object of a webpage
 \item[N]: the number of inline objects of a webpage
 \item[O]: the size of inline objects
 \item[R]: the size of \textit{GetRequests}

\end{description}
Thus being said, the random variable are distributed in the following way:

\begin{description}
 \item[I] - it is hard to establish a distribution, so it is used I = 1 second between two sessions;
 \item[X] - Lognormal distributed (\ref{lognormal_dist}) with a mean E = 25 and a standard deviation $\sigma$=100;
 \item[V] - Weibull distributed (\ref{weibull_dist}) with a mean of 39.5 seconds( $\alpha$ = 0.5, $\beta$ = 4.44);
 \item[M] - Lognormal distributed(\ref{lognormal_dist}) ($\alpha$ = 1.31, $\beta$ = 1.41) - with a mean of 10kB;
 \item[N] - Gamma distributed(\ref{gamma_dist}) ($\alpha$ = 0.24, $\beta$ = 23.42) - with a mean of 5.55 inline objects per webpage;
 \item[O] - Lognormal distributed(\ref{lognormal_dist}) ($\alpha$ = -0.75, $\beta$ = 2.36) - with a mean of 7.7kB;
 \item[R] - Lognormal distributed(\ref{lognormal_dist}) ($\alpha$ = 5.84, $\beta$ = 0.29) - with a mean of 360B;
\end{description}

\subsection{Implementation of the simulator}
At first, we describe a general simulator:
\begin{verbatim}

   // We create a general simulator
       motSim_create();          
       sink = PDUSink_create();
   // The recommended model uses for interarrival time a Weibull distribution
  dateGenWb = dateGenerator_createWeibull(alpha, beta);
  
  server = srvGen_create(sink, (processPDU_t)PDUSink_processPDU);
  srvGen_setServiceTime(server, serviceTimeProp, 1.0/debit); 

  filePDU = filePDU_create(server, (processPDU_t)srvGen_processPDU); 

\end{verbatim}

\subsubsection{The request}

The variable V described before represents the time when a user reads a page, the time between two consecutive requests.
 This being told, for the request part, I suggest the following code to set the simulator:

\begin{verbatim}

   /* The size of GetRequest packets is Lognormal distributed 
      with req_alpha = 5.84 and req_beta = 0.29
   */    
      getReqSzGen = randomGenerator_createDouble();
      randomGenerator_setDistributionLognormal(getReqSzGen, req_alpha, req_beta);
      
      getRequest = PDUSource_create(dateGenWb, filePDU, (processPDU_t)filePDU_processPDU); 
    /* We associate the PDUSource the random size generator getReqSzGen */
      PDUSource_setPDUSizeGenerator(reqPDU, getReqSzGen);

\end{verbatim}

 To start the simulator:
\begin{verbatim}
      PDUSource_start(getReqSzGen);
\end{verbatim}

\subsubsection{The Reply} 

The random variables for main object size, inline object average size and number use Lognormal and Gamma distributions.
 These are implemented in the source random-generator.c

To set the simualtor for a http reply:

\begin{verbatim}
      reqSizeGen = randomGenerator_createUInt();
      randomGenerator_setDistributionComposed(reqSizeGen, a, b, ain, bin, galpha, gbeta);
      
 
      reqPDU = PDUSource_create(dateGenWb, filePDU, (processPDU_t)filePDU_processPDU);  
  
      /* We associate the size of the request packet to the reqPDU */
      PDUSource_setPDUSizeGenerator(reqPDU, reqSizeGen);
\end{verbatim}

 To start the simulator:
\begin{verbatim}
PDUSource_start(reqPDU);
\end{verbatim}
 
  For the structure of the page, I used a random generator composed from three distributions(\ref{composed_rg}) for the random variables: main object size, inline object size and number of inline objects.
  In the distribution parameter structure of the random generator, I added also a the option of a size main generator, inline main generator and a number inline object generator.

\begin{verbatim}
         struct randomGenerator_t * szmain;
         struct randomGenerator_t * szin;
         struct randomGenerator_t * nin;
\end{verbatim} 




\section{A FTP model}

The model implemented here is recommended in "cdma2000 Evaluation Methodology".

The structure of a FTP session is:
\begin{verbatim}

  struct FTP_Session_t{
    declareAsNdesObject;
    struct dateGenerator_t    *readingTime;      //!< the time between two ftp sessions
    struct randomGenerator_t  *sizeGen;          //!< the random generator for the size of files

    struct PDUList_t  *firstPdu;         //!< the first pdu in the list that need to be transmitted
    struct PDUList_t  *lastPdu;          //!< the last pdu in the list that need to be transmitted
    int PDUnr;                           //!< the number of PDUs in the list
    double duration;                     //!< the duration of a session

    void *destination;                    //!< the destination of the files
    processPDU_t destProcessPDU;          //!< a pointer to the function that will process the files

    int SessionID;                        //!< each session has a different ID
};

\end{verbatim}

The important parameters for us are:
\begin{description}
\item[readingTime] - the time between two consecutive sessions and it is a random variable
based on an Exponential distribution with $\lambda=0.006$.
\item[sizeGen] - the size of each file that will be transferred, it's a random variable based on a Lognormal distribution \ref{lognormal_dist} (with $\alpha=14.45$ and $\beta=0.35$).
\end{description}


To create a simulation, we simply call the function:
\begin{verbatim}
struct FTP_Session_t* FTP_Session_Create(struct dateGenerator_t *dateGen,
                                          struct randomGenerator_t *sizeGen,
                                          void *destination,
                                          processPDU_t destProcessPDU)
\end{verbatim}

\textbf{dateGen} and \textbf{sizeGen} are the parameters I described above.
\textbf{destination} is the destination of the files and \textbf{destProcessPDU} is the function that will process the files.

The algorithm is even simpler than the one from http model:
\begin{enumerate}
\item When the session starts, we have all the parameters we need to transmit a file, so we split the file into multiple PDUs of maximum 1460bytes (MSS = MTU-40bytes).
\item We transmit all the PDUs in order
\item We prepare the parameters for the next session and we add a new event session to the simulator
\end{enumerate}
The algorithm repeats until a preseted time expires.

\subsection{Implementing a simple ftp simulation}

At first, we describe a general simulator:

\begin{verbatim}
 motSim_create();
 sink = PDUSink_create();
\end{verbatim}
The sink is the destination of the transmitted file at the end of a transfer.

Because between a client and a ftp server is usually some other device (in wireless networks, this is called base station, in local area networks, it is called router), we created a node of the type srvGen\_t:
\begin{verbatim}
base_station = srvGen_create(sink, (processPDU_t)PDUSink_processPDU );
srvGen_setServiceTime (base_station, serviceTimeExp, mean);
\end{verbatim}
 The destination of packets that the router/base station forwards is the sink created before.
 This node serves each packet in exponential time, with mean = 50ms. Before creating the base station, we set the mean variable equal to 50.

We need to define a filePDU what will hold the created PDUs and from which we will take one PDU at a time to send to the base station:
\begin{verbatim}
filePDU = filePDU_create(base_station, (processPDU_t)srvGen_processPDU);
\end{verbatim}

Then we set the parameters for the transfer:
\begin{verbatim}
    fileSize = randomGenerator_createDouble();
    randomGenerator_setDistributionLognormal(fileSize, 14.45, 0.35);
  
    readingTime = dateGenerator_createExp(0.006);
\end{verbatim}

Now that we have initialised all the structure, we can create the one that intereses us more: the ftp session's structure that holds all the parameters for the transfer.
\begin{verbatim}
Transfer = FTP_Session_Create(readingTime, fileSize, filePDU, (processPDU_t) filePDU_processPDU );
\end{verbatim}

We can also put some sensors on the simulator:
\begin{verbatim}
   // A probe for the inter-arrival files
    isProbe = probe_createExhaustive();
    dateGenerator_addInterArrivalProbe(readingTime, isProbe);
   // A probe for the base station's service time
    bsProbe = probe_createExhaustive();
    srvGen_addServiceProbe(base_station, bsProbe);
   // A probe for the journey 
    sejProbe = probe_createExhaustive();
    filePDU_addSejournProbe(filePDU, sejProbe);
\end{verbatim}

 We are ready to start the simulator. We start the sessions:
\begin{verbatim}
  FTP_Session_Start(Transfer);
\end{verbatim}
and the simulator:
\begin{verbatim}
    FTP_Session_Start(Transfer);

    motSim_runUntil(duration);
\end{verbatim}
 To see the simulation status
\begin{verbatim}
 motSim_printStatus();
\end{verbatim}
And we can also see the results of the accumulated probes:
\begin{verbatim}
    printf("Number of FTP sessions : %d \n", FTP_GetSessionNr(Transfer));
    printf("Average reading time: %f ms \n", probe_mean(isProbe));
    printf("Base station serving time: %f \n", probe_mean(bsProbe));
    printf("Average journey time: %f \n", probe_mean(sejProbe));
\end{verbatim}









%========================================================================
%
%========================================================================


\chapter{The library}

%\input{sondes-en}
%========================================================================
%
%========================================================================
\section{The random generators}
\label{section:random-generator}

  In a simulator, the generation of random numbers is an important part. 
In {\sc ndes}, we take this very serious. The random number generator 
management is a nameless horror! I admit I had problems understanding it too. 
The good knews is that the result is really \ldots{} random.

%------------------------------------------------------------------------
%
%------------------------------------------------------------------------
\subsection{The characteristics of a random generator}

 A random generator is characterised by three important components:

\begin{description}
   \item[The type of generated data] can be real, integer,
     discrete, \ldots
   \item[The law] can be uniform, exponential, \ldots
   \item[The source] permits to determine the quality of the randomness, and 
by example to render it deterministic (to obtain the same sequence with other 
simulation).
\end{description}

  Be careful, these three components eventualy use some parameters.
  
    The general principle for generating a variable is described below.

\begin{itemize}
   \item A random number is provided by the source. It will be a
      integer between two extreme values​​, for example, depending on
      the nature of the source.
   \item A transformation is applied to meet the probability density 
      of the law.
   \item A second transformation is applied to obtain a
      value of the desired type.
\end{itemize}

   Shortly, everything is done to make the generation random \ldots


%------------------------------------------------------------------------
%
%------------------------------------------------------------------------
\subsection{The use}

  The general schema for using the generator is simple: we create a generator,
we initialise with the desired values, we associate with it a probability 
distribution, eventualy we can change the source of randomness on which it is based,
then we can extract the random values and at the end we destroy the generator.

   Observe in detail relevant functions of this exciting program.

%------------------------------------------------------------------------
%
%------------------------------------------------------------------------
\subsection{The creation and the distruction}
   
   There is at least a function of creation for each type of
managed data (it fails more than that!). But sometimes, there are also
functions to specify the distribution to use at the same time. 
See this type per type.

%.......................................................................
%
%.......................................................................
\subsubsection{The unsigned integers}

   The function for creating the base is 

\index{randomGenerator\_createUInt}
\begin{verbatim}
struct randomGenerator_t * randomGenerator_createUInt();
\end{verbatim}

%.......................................................................
%
%.......................................................................
\subsubsection{The unsigned integeres between {\tt min} and {\tt max}}

   It is fun to play dice. This can be created with

\begin{verbatim}
struct randomGenerator_t * randomGenerator_createUIntRange(unsigned int min,
						      unsigned int max);
\end{verbatim}

%.......................................................................
%
%.......................................................................
\subsubsection{A list of unsigned integers}

   Practice to randomize packet sizes!

\index{randomGenerator\_createUIntDiscrete}
\begin{verbatim}
struct randomGenerator_t * randomGenerator_createUIntDiscrete(int nbValues,
							      unsigned int * values);
\end{verbatim}

   The first parameter gives the number of values, and the second is 
a table which contains (at least) these values. Its contents will be
copied, so if you want to destroy / modify then, you're free to do it!
   As there is little doubt that in such situations we will
want to assign a probability to each value, you can use the
following version:

\index{randomGenerator\_createUIntDiscreteProba}
\begin{verbatim}
struct randomGenerator_t * randomGenerator_createUIntDiscreteProba(
				int nbValues,
				unsigned int * values,
				double * proba);
\end{verbatim}

%.......................................................................
%
%.......................................................................
\subsubsection{The real numbers in double precision}

   We create a generator with the following function

\index{randomGenerator\_createDouble}
\begin{verbatim}
struct randomGenerator_t * randomGenerator_createDouble();
\end{verbatim}

   We may also create one generator based on an exponential distribution
parameter {\tt lambda} :

\index{randomGenerator\_createDoubleExp}
\begin{verbatim}
struct randomGenerator_t * randomGenerator_createDoubleExp(double lambda);
\end{verbatim}

%.......................................................................
%
%.......................................................................
\subsubsection{The real numbers in double precision between {\tt min} and {\tt max}}

\begin{verbatim}
\end{verbatim}

%.......................................................................
%
%.......................................................................
\subsubsection{A list of real numbers in double precision}

   For generating random numbers from a list with provided
parameters:

\index{randomGenerator\_createDoubleDiscrete}
\begin{verbatim}
struct randomGenerator_t * randomGenerator_createDoubleDiscrete(
                                     int nbValues,
                                     double * values);
\end{verbatim}

   or, providing the probabilities directly:

\index{randomGenerator\_createDoubleDiscreteProba}
\begin{verbatim}
struct randomGenerator_t * randomGenerator_createDoubleDiscreteProba(
                                     int nbValues,
                                     double * values,
                                     double * proba);
\end{verbatim}

   The parameters are similar to the release based on unsigned short 
int, according on what we see above.

%.......................................................................
%
%.......................................................................
\subsubsection{The distruction}

   We destroy the generator with the function

\index{randomGenerator\_delete}
\begin{verbatim}
void randomGenerator_delete(struct randomGenerator_t * rg);
\end{verbatim}

%------------------------------------------------------------------------
%
%------------------------------------------------------------------------
\subsection{Choosing the law}

   Before it can be used, a random generator need to be characterised
by a law who governs it. There are specific functions for this purpose. 
Some function of creation call either of these functions, but not all! 
So be careful to ensure that a distribution is associated with a generator 
before using it.

%.......................................................................
%
%.......................................................................
\subsubsection{Uniform law}

   We specify a uniform law through the following function

\index{randomGenerator\_setDistributionUniform}
\begin{verbatim}
void randomGenerator_setDistributionUniform(struct randomGenerator_t * rg);
\end{verbatim}
   
   Be careful, the type of data can be a "continuous" bounded interval
or a discrete set, but if it is an unbounded interval, the result is
\ldots {} random.

%.......................................................................
%
%.......................................................................
\subsubsection{Explicit law}
   
   I do not know how to call that! The idea is that it explicitly provides 
all the probabilities. It is specified by the following function:
   
\index{randomGenerator\_setDistributionDiscrete}
\begin{verbatim}
void randomGenerator_setDistributionDiscrete(struct randomGenerator_t * rg,
					     int nb,
                                             double * proba);
\end{verbatim}
   
   Attention, it obviously only applies to discrete data!

   The probabilities are copied by the functions so the pointer
{\tt proba} can be released.

%------------------------------------------------------------------------
%
%------------------------------------------------------------------------
\subsection{Choosing the source}

%------------------------------------------------------------------------
%
%------------------------------------------------------------------------
\subsection{Generating a value}

\index{randomGenerator\_getNextUInt}
\index{randomGenerator\_getNextDouble}

   A new value is obtained at each call of the following functions 
(chosen according to the expected type)

\begin{verbatim}
unsigned int randomGenerator_getNextUInt(struct randomGenerator_t * rg);
double randomGenerator_getNextDouble(struct randomGenerator_t * rg);
\end{verbatim}

%------------------------------------------------------------------------
%
%------------------------------------------------------------------------
\subsection{The sondes}


%\input{date-generator-en}
%\input{pdu-en}
%%========================================================================
%
%========================================================================
\section{Les files de {\sc pdu}}
\label{section:file}
  

   This is the basic tool for storing objects using a {\sc fifo} strategy. They are defined in the file {\tt file\_pdu.h}
and located in the file {\tt file\_pdu.c}. A file doesn't contain only
{\sc pdu}s, but as we have seen, the {\sc pdu}s can be
used to convey everything and anything through their private data. It is simple to construct files of anything!

    By default, the size of a file is limited by the capacity of the
system.
  
%------------------------------------------------------------------------
%
%------------------------------------------------------------------------
\subsection{create a file}

   A file is created in the following way:

\index{filePDU\_create}
\begin{verbatim}
struct filePDU_t * filePDU_create(void * destination,
			    processPDU_t destProcessPDU);
\end{verbatim}

   The parameter \lstinline!destination! is a pointer of an object 
which transmits the objects in the file. As soon as a
object is inserted into the queue, if the destination is available,
is sent through the function \lstinline!destProcessPDU! passed as parameter.

%------------------------------------------------------------------------
%
%------------------------------------------------------------------------
\subsection{Inserting a file}

   
   We insert a file in a queue with a standard function:
  

\index{filePDU\_insert}
\begin{verbatim}
void filePDU_insert(struct filePDU_t * file,
		    struct PDU_t * PDU);
\end{verbatim}

   The input-output function: 

\index{filePDU\_processPDU}
\begin{verbatim}
int filePDU_processPDU(void * file,
		       getPDU_t getPDU,
		       void * source);
\end{verbatim}

   It allows you to use the file like any other entity
in a simulator (see for example the first tutorial).

%------------------------------------------------------------------------
%
%------------------------------------------------------------------------
\subsection{Extracting a file}

   The extraction of a {\sc pdu} from a file:

\index{filePDU\_extract}
\begin{verbatim}
struct PDU_t * filePDU_extract(struct filePDU_t * file);
\end{verbatim}

   where the return value can be {\tt NULL} in case of empty file.

   Here too, a specific function of the simulation is provided:

\index{filePDU\_getPDU}
\begin{verbatim}
struct PDU_t * filePDU_getPDU(void * file);
\end{verbatim}

%------------------------------------------------------------------------
%
%------------------------------------------------------------------------
\subsection{Gestioning the size}

   Two functions allow you to view the size of the file, the
first gives the size in number of {\sc pdu}s.

\index{filePDU\_length}
\begin{verbatim}
int filePDU_length(struct filePDU_t * file);
\end{verbatim}

   the second gives the cumulative sum of the sizes of {\sc pdu}s
present:

\index{filePDU\_size}
\begin{verbatim}
int filePDU_size(struct filePDU_t * file);
\end{verbatim}

%........................................................................
%
%........................................................................
\subsubsection{Maximum size}

   It is possible to limit the size of a file. The upper bound
can be expressed as a number of {\sc pdu}s or presented as 
cumulative size:

\index{filePDU\_setMaxSize}
\index{filePDU\_setMaxLength}
\begin{verbatim}
void filePDU_setMaxSize(struct filePDU_t * file, unsigned long maxSize);
void filePDU_setMaxLength(struct filePDU_t * file, unsigned long maxLength);
\end{verbatim}

   A value of zero means no limit (this is the default
default). Both parameters can be used simultanéments
(it is then the more restrictive blocking). Their value can be
accessed through the following functions:

\index{filePDU\_getMaxSize}
\index{filePDU\_getMaxLength}
\begin{verbatim}
unsigned long filePDU_getMaxSize(struct filePDU_t * file);
unsigned long filePDU_getMaxLength(struct filePDU_t * file);
\end{verbatim}

%........................................................................
%
%........................................................................
\subsubsection{Et en cas de débordement ?}

%------------------------------------------------------------------------
%
%------------------------------------------------------------------------
\subsection{The sensors}

   Les files sont dotées des points de mesure suivants

\begin{description}
   \item[{\tt InsertSize}] pour mesurer la taille des paquets insérés
     dans la file. A chaque insertion d'une {\sc pdu}, la taille de
     cette dernière est échantillonée avec la date d'insertion.
   \item[{\tt ExtractSize}] pour mesurer la taille des paquets extraits
     de la file. A chaque extraction d'une {\sc pdu}, la taille de
     cette dernière est échantillonée avec la date d'extraction.
   \item[{\tt DropSize}]
   \item[{\tt Sejourn}]
\end{description}

   On peut ajouter une sonde sur chacun de ces points de mesure avec
une fonction de la forme

\begin{verbatim}
   void filePDU_add<measure>Probe(struct filePDU_t * file,
                                  struct probe_t   * probe);
\end{verbatim}


%\input{srv-gen-en}
%\input{pdu-sink-en}
%\input{link-simplex-en}
%\input{dvbs2link-en}
%\input{sched-acm-en}
%\input{sched-acm-knapsack-en}


\section{GnuPlot diagrams}
\label{section:gnuplot}

%========================================================================
%
%========================================================================

\section{Notion of simulation and of campaign}

A simulation is a single instance of an execution of an event sequence
following the initialization of the model. A campaign is a
result of simulations on the same model with a reset of
variables between two simulations.

    The sensors relating to the simulation are re-initialized at the end of the
simulation. Probes may be related to the campaign; they will not be reseted at the end of the campaign and they can allow also inter-simulation values​​, for example confidence intervals.

%========================================================================
%
%========================================================================
%========================================================================
%
%========================================================================
\chapter{Annexes}

\section{Function index}

\printindex

\end{document}
