%========================================================================
%
%========================================================================
\section{General Introduction}

   {\em Network Discrete Event Simulator} (or {\sc ndes}) is a library
that permits the creation of discrete event simulators. This library can be used for different purposes. We classify individuals into two categories according to
their goals.

\begin{description}
   \item[An user] is a person who wishes to write a
      simulation program based on this library. He only needs
      the compiled version of this library. It is not
      necessary for him to know the details of the operation.

   \item[A programmer] is a person who wants to improve and enrich
      or debug the library. So he needs to handle the
      source code, and for that, to understand the operation.
\end{description}

   The difference between these two is important for this document.

   The user can start with reading the section
\ref{section:user-tuto} who presents the use of a simulator in a tutorial 
(whose code is included with the {\sc ndes} source). He can also consult the sections that presents the tools of the simulator (the sensors, the generators, \ldots) and the library of models.

   The programmer, meanwhile, should read the dedicated extended section of the simulator, as shown here in the form of a tutorial, then he should look at the sections describing the complications of {\sc ndes}, by example, the simulation engine.

   
   As it often happens in this kind of project, unfortunately, it is
unrealistic to expect a documentation update. The primary objective is
developing any tool in order to improve the targeted use of that
maintain documentation. The reader is invited to consult
the following sources of information, in reverse order of chronology:


\begin{itemize}
   \item this document ;
   \item extracted documentation of sources, presented
      in the directory {\tt DOCS} ;
   \item the source code.
\end{itemize}

