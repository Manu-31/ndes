%------------------------------------------------------------------------
%
%------------------------------------------------------------------------
\section{Un ordonnanceur {\sc acm}}

   Les fichiers {\tt schedACM.h} et {\tt schedACM.c} définissent et
implantent un cadre général et des fonctions communes permettant de
construire des algorithmes d'ordonnancement pour des liens implantant
des techniques de type {\sc acm} ({\em Adaptive Coding and
  Modulation}).

   L'objectif de cette structure est de faire des choses aussi
génériques que possible qui puissent être adaptées à différents
contexte. Ce n'est pas encore complètement le cas et ces fichiers
restent relativement dépendants de DVB-S2.

%.......................................................................
%
%.......................................................................
\subsection{Ajout d'une file}

%.......................................................................
%
%.......................................................................
\subsection{Principe général}

   Le principe général est le suivant.

\begin{itemize}
   \item L'ordonnanceur va chercher une ``solution'' par le biais d'une
   fonction {\tt schedule}.
   \item Une trame va être construite en fonction de cette solution
     par la fonction {\tt buildBBFRAME}.
   \item Cette trame sera transmise au travers de la fonction {\tt
     getPDU}.
\end{itemize}

   Cette structure est assez riche car une ``solution'' peut être
extrèmement complexe. Nous envisageons en effet ici le cas le plus
général dans lequel il est éventuellement possible de transmettre
plusieurs paquets dans une même trame.

   De ce fait, une ``solution'' sera décrite par une structure
permettant cette souplesse.

%.......................................................................
%
%.......................................................................
\subsection{Les fonctions spécifiques à un ordonnanceur}

   Chaque type d'ordonnanceur {\sc acm} pourra implanter un certain
nombre de fonctions spécifiques. Elles seront enregistrées dans une
structure du type suivant 

\index{struct schedACM\_func\_t}
\begin{verbatim}
struct schedACM_func_t {
   struct PDU_t * (*getPDU)(void * private);
   int  (*processPDU)(void * private,
	               getPDU_t getPDU, void * source);

   struct PDU_t * (* buildBBFRAME)(void * private);

   void (*schedule)(void * private);
};
\end{verbatim}

   Les trois premières sont optionnelles, car une fonction par défaut
est définie qui devrait convenir dans la plupart des cas. C'est la
dernière qui implante l'algorithme lui-même, elle est donc nécessaire.

\paragraph{La fonction {\tt getPDU}}  est la fonction qui sera invoquée pour
demander à l'ordonnanceur de ``produire'' une {\sc pdu} (en
ordonnançant des paquets). Si elle n'est pas définie, la fonction
\index{schedACM\_getPDUGeneric}{\tt schedACM\_getPDUGeneric} sera
utilisée par défaut. Cette fonction se contente de faire appel à la
fonction {\tt buildBBFRAME}.

\paragraph{La fonction {\tt buildBBFRAME}} a pour rôle de construire
une trame avec les paquets choisis par l'ordonnanceur. Elle renvoie
donc un pointeur vers la {\sc pdu} ains créée. En l'absence d'une
telle fonction définie, c'est la fonction
\index{schedACM\_buildBBFRAMEGeneric} {\tt
  schedACM\_buildBBFRAMEGeneric} qui sera invoquée.

   Nous décrirons cette fonction plus loin.

%.......................................................................
%
%.......................................................................
\subsection{Définition des solutions}

   Nous l'avons dit, une solution trouvée par l'ordonnanceur peut être
composés de plusieurs paquets, provenant éventuellement de
plusieurs files.

   La structure permettant de modéliser cela est la suivante

\index{t\_remplissage}
\begin{verbatim}
typedef struct {
   int       modcod;         //!< Le numero de MODCOD choisi
   int    ** nbrePaquets;    //!< Nombre de paquets de chaque file à transmettre
   int       volumeTotal;    //!< Le nombre total d'octets à transmettre

   /* Les champs suivants sont à la dispo de l'ordonnanceur. Il
      faudrait surement faire plus propre, avec un  pointeur sur prive
      ou une union, ... */
   double    interet;
   int       nbChoix;        // Nombre de choix menant à cet interet
   int       casTraite;      // Pour éviter de retraiter un cas
} t_remplissage ;
\end{verbatim}


%.......................................................................
%
%.......................................................................
\subsection{La fonction {\tt schedACM\_buildBBFRAMEGeneric}}
\index{schedACM\_buildBBFRAMEGeneric|(}

   Cette fonction s'occupe de consulter la solution fournie par
l'ordonnanceur, de construire la trame correspondante et d'extraire
les paquets à transmettre des files d'entrée.

\index{schedACM\_buildBBFRAMEGeneric|)}
