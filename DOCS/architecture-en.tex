%========================================================================
%
%========================================================================
\section{General architecture}
\label{section:architecture}

   
The purpose of this section is to describe the logic of {\sc
ndes}.

%------------------------------------------------------------------------
%
%------------------------------------------------------------------------
\subsection{Type of the simulation}

   {\sc Ndes} is a library of a discrete event network
simulator. No, this is not original, but it is still
really important for the future. All the code for processing a
event will be executed into a long time, but absolutely not recognized in simulated time. And worse! During the processing of an event, such as time
simulated is fixed, the system does not change (except through the code in
question). 

%------------------------------------------------------------------------
%
%------------------------------------------------------------------------
\subsection{General schema of the models}

   The basic idea is that the simulated network can be
modeled as a sequence of entitées chained together. In the
strictly linear simplest version, the first of these
entities will produce messages, called {\sc pdu}s, it will
follow the following entity and so on until the last one
may conserve the messages or destroy them.

   The M/M/1 queue described in the first tutorial is a perfect
example of this model. We already know the four types of
nodes: {\tt sourcePDU}, {\tt filePDU}, {\tt serv\_gen} and {\tt PDUSink}.

   The construction of the network is starting "later" and
based on the source, as it can be seen in the
tutorial. Each time you create an entity, you must pass
parameters as information about the downstream entity. 
However, we don't need to know anything about these entities. Yes, it
there can be several.

   Some entities may also have multiple downstream entities
(to be provided by a specific function), but the {\sc pdu}
data is passed to only one of them.

   
    The bad news is that this scheme will be reviewed soon
or later! It is not good to model things more
symmetric (a protocol entity that sends and receives), and
things to be reviewed as the term {\sc pdu} is not at all
appropriate. It is just {\sc pdu} which is represented by
this thing!

%------------------------------------------------------------------------
%
%------------------------------------------------------------------------
\subsection{The exchange {\sc pdu} function}

   Because of the general network architecture for any entity capable
generating or transferring PDU must be a function of the form

\begin{verbatim}
   struct PDU_t * getPDU(void * source);
\end{verbatim}

   The parameter is a pointer to the "private data" for
identify the node(typically a direct pointer to the node).

   The returned pointer is a PDU which is used by the source. 
It must absolutely be managed (or, less, destroyed) by the user of this function. 
If PDU is unavailable, the {\tt NULL} value is returned.

   This function and the associated pointer must be provided to
the recipient entity, if any!

    If the module name is {\tt foo}, the function will be named for example
{\tt foo\_getPDU().}\\

    Symmetrically, any module that may receive PDU must provide
function of the form:

\begin{verbatim}
   int processPDU(void * rec,
                  getPDU\_t getPDU,
                  void * source);
\end{verbatim}

   
    It is this function who invokes a source to notify the
availability of a PDU. This function will have the responsibility
go get the PDU (through the function \lstinline!getPDU! and the source included) 
and to treat it. The recovery and treating may
be postponed (if unavailable) but there is a risk to 
have a \lstinline!NULL! pointer returned by
\lstinline!getPDU()!.

   The value returned by this function is zero in case of failure, and
non-zero if successful.

    This function can be invoked with the last two parameters
{\tt NULL}. In this case, the only purpose is to determine if {\tt rec}
entity is ready to process a new {\sc pdu}. It
must return 0 if it is occupied or 1 if it is free.

   This property must be implemented in all cases and may
for example be used by an upstream entity that can keep
{\sc pdu}s (a file, for example) to avoid loss due to
a downstream entity that could not receive a new {\sc pdu} (a
link during transmission, for example).

   An example of utilisation is given in the first ''{\em Programmer's tutorial}'' in sub-section \ref{subsection:tut-ordo}.

   If the module name is {\tt foo}, the function will be named for example
{\tt foo\_processPDU()}.

    An then you tell me ``what is this site?!''. In fact, that's how it's supposed to work (and it looks like it's working!). Imagine two entities {\tt A} and {\tt B} of our
network, {\tt B} being the downstream of {\tt A} as in Figure \ref{figure: reseauAB}.

%.......................................................................
%
%.......................................................................
\subsubsection{The transmision of a {\sc pdu}}

   
   If, while processing an event, the entity {\tt A} must
follow a {\sc pdu} to {\tt B}, it will do by invoking the function
\lstinline!processPDU! associated with {\tt B}. From that moment, the
{\sc} pdu is the responsibility of {\tt B}. Two cases are
possible:

\begin{itemize}
   \item Either {\tt B} is ready to process the {\tt pdu} (this is a file
      not full, or an inactive server), then it invokes
      immediately (within its code \lstinline!processPDU!) the
      function \lstinline!getPDU! of {\tt A} and all is good.
   \item Either {\tt B} is not ready. In the code of the
      \lstinline!processPDU!, there will be no invocation of
      \lstinline!getPDU! of {\tt A}, but there must be an action
      to do it later \footnote{In fact, it is not
      quite true! We must be sure that {\tt B} will
      recover the {\tt pdu}, but we can make other arrangements
      as we will see with the arrival of a {\sc pdu}} (in the simulated time!). 
      But for that invocation to happen, perhaps {\tt A} has destroyed the {\sc pdu} 
      (guess {\tt A} models a physical layer, it
      will not hold a {\sc pdu}. Finally, it's reasonable!
\end{itemize}

%.......................................................................
%
%.......................................................................
\subsubsection{The arrival of a {\sc pdu}}

   On the other hand, if, during the processing of an event, the entity
{\tt B} is ready to consume a {\sc pdu}, it will by invoking
the function \lstinline!getPDU! associated with {\tt A}.

    Again, two cases have been recognised:

\begin{itemize}
   \item Either {\tt A} has actually a {\sc pdu} to provide
      to {\tt B}, and in this case everything goes well!
   \item Either {\tt A} doesn't have {\sc pdu}s. In that case, the function
     \lstinline!getPDU! will transmit a \lstinline!NULL! pointer who
     will be ready to treat.
\end{itemize}

   The invocation of the function \lstinline!GetPDU! can be triggered by an event that has nothing to do with it, this is how we will get the {\sc pdu}s without
having been requested, where the previous note, and (among other things) it 
risks a \lstinline!NULL! pointer.

%.......................................................................
%
%.......................................................................
\subsubsection{The chronology of the events}

   All of this may seem a bit vicious, and you may be unclear to know what 
function to use when! But do not worry, all that has to be mastered by anyone 
who wants create new types of entities in the simulator.

    On the other hand, it is actually very simple, just follow the
logic of events. For example, imagine that you want to
model a {\em round-robin} scheduler. We will assume that it is
downstream of a number of files, and upstream of a server that
models a communication link. See Figure
\ref {figure: exempleordo}.    

% Figure

%
%
%
\paragraph{\lstinline!schedRRProcessPDU!}

   How it will look like the code for the treatment of a {\sc pdu} in this scheduler?

    This function is invoked when one of the upstream queues will have a
{\sc pdu} to provide to the scheduler. But the {\sc pdu} must
be scheduled only if the following two conditions are satisfied:
\begin{itemize}
   \item the support (downstream) is free;
   \item it is the turn of the queue to be served or the other is
      empty \footnote{If not it will not work conserving}.
\end{itemize}

   The function \lstinline!SchedRRProcessPDU! will therefore have to test these
conditions and, if they are true, actually retrieve the {\sc pdu} (with \lstinline!getPDU! call) and then to send the
link (with \lstinline!processPDU! of the server).

    If the conditions are not satisfied, it leaves the {\sc pdu} where it is.

%
%
%
\paragraph{\lstinline!schedRRGetPDU!}

   And now to whom it looks like this function to obtain a {\sc pdu} our scheduler?

   It is invoked by the communication support(downstream) when
it is free (an event of the end of the transmission from the preceding
example).

   The scheduler does not have {\sc pdu}s itself. It should
pick up in queues upstream next to him and serve
a {\sc pdu} through his \lstinline!getPDU!. If
it finds anything, it returns a \lstinline!NULL! pointer to downstream entity
(link) which therefore does nothing.

   In this situation, the next event is the arrival of a
{\sc pdu} in a queue by invoking the \lstinline!processPDU! function
that invoke the scheduler, \ldots

%------------------------------------------------------------------------
%
%------------------------------------------------------------------------
\subsection{Several transmitters to the same recipient}

   How about when several entities are upstream of the same
entitée downstream? For example, several sources that send customers
to a single server.

   In fact, this is not a problem, the downstream entity does not generally understand the entity upstream . So, if there are two or more, it doesn't care. 
Yes, except no! Imagine an entity that is not always prepared to deal with an
incoming {\sc pdu}. In general, it will store somewhere a
information that will allow him to pick the {\sc pdu} when it wants. 
If this happens several times before the entity is willing to finally treat the waiting{\sc pdu}s, only the last may have been stored !

   So, we might be tempted to make a linked list of
events, so as not to miss any . Unfortunately , this puts
necessarily a policy {\sc FCFS }, then we want to implement something else.

