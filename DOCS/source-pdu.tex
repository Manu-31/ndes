%========================================================================
%
%========================================================================
\section{Les sources de {\sc pdu}}
\label{section:pdu-source}

   C'est bien joli d'avoir des {\sc pdu}, mais dans un réseau, il faut
au moins une entité pour en injecter, sinon ça risque d'être un peu
triste ! C'est le rôle des sources de {\sc pdu}. Ce sont elles qui
vont injecter le trafic dans le réseau.

   Le type le plus général est fourni dans {\tt
include/pdu-source.h}. Il s'agit d'un outil relativement élémentaire
   mais particulièrement simple.

%-------------------------------------------------------------------------
%
%-------------------------------------------------------------------------
\subsection{La source de base}

   Le modèle de source le plus simple est donc décrit dans {\tt
include/pdu-source.h} et implanté dans {\tt src/pdu-source.c}. Le type
   suivant est défini 

\index{struct PDUSource\_t}
\begin{verbatim}
   struct PDUSource_t;
\end{verbatim}

   Une source sera essentiellement caractérisée par deux jeux de
composantes 

\begin{itemize}
   \item la loi régissant la date d'émission des {\sc pdu} ;
   \item la loi régissant la taille des {\sc pdu}.
\end{itemize}

%........................................................................
%
%........................................................................
\subsubsection{Création d'une source}

   Une source est créée grâce à la fonction suivante

\index{PDUSource\_create}
\begin{verbatim}
struct PDUSource_t * PDUSource_create(struct dateGenerator_t * dateGen,
				      void * destination,
				      processPDU_t destProcessPDU);
\end{verbatim}

   Les deux derniers paramètres caractérisent l'entité à laquelle
doivent être transmises. Il s'agit, respectivement, d'un pointeur vers
cette entité et  de la fonction par laquelle elle traite les {\sc pdu}
entrantes.

   Le premier paramètre est un pointeur vers un générateur de dates
qui sera utilisé pour déterminer les dates  de production des {\sc
  pdu} consécutives.

   Par défaut, la taille des {\sc pdu} ainsi produite est nulle.

%
%
%
\paragraph{Source déterministe}

   Il est également possible de créer une source ``déterministe'',
c'est-à-dire dont les lois de création et de taille des paquets soient
décrites par une liste exhaustive de valeurs prédéfinies.

   On utilisera pour cela la fonction suivante

\index{PDUSource\_createDeterministic}
\begin{verbatim}
struct PDUSource_t * PDUSource_createDeterministic(struct dateSize * sequence,
						   void * destination,
						   processPDU_t
destProcessPDU);
\end{verbatim}

   Le permier pointeur est un tableau de couples date/taille terminé
par un couple nul. La structure \lstinline!struct dateSize! est
définie de la façon suivante 

\index{struct dateSize}
\begin{verbatim}
struct dateSize {
   motSimDate_t date;
   unsigned int size;
};
\end{verbatim}

   À titre d'exemple, le programme {\tt examples/inoutdemo.c} défini
une séquence de trois {\sc pdu} de tailles respectives 1000, 2000 et
2000 qui sont émises aux dates 0.0, 0.5 et 1.0. Cette définition est
faite de la façon suivante

\begin{verbatim}
   struct dateSize sequence[] = {
      {0.0, 1000.0},
      {0.5, 2000.0},
      {1.0, 2000.0},
      {0.0, 0.0}
   };
\end{verbatim}

%........................................................................
%
%........................................................................
\subsubsection{Choix de la loi d'arrivée}

   La loi régissant la date de production des {\sc pdu} peut être
sélectionnée grâce à la fonction suivante

\index{PDUSource\_setDateGenerator}
\begin{verbatim}
void PDUSource_setDateGenerator(struct PDUSource_t * src,
                                struct dateGenerator_t * dateGen);
\end{verbatim}

   Les paramètres sont clairs ! {\tt src} pointe sur la source à
modifier et {\tt dateGen} est un pointeur sur le générateur de dates à
lui associer. Les générateurs de dates sont décrits dans la section
\ref{section:date-gen}.

%........................................................................
%
%........................................................................
\subsubsection{Choix de la loi de la taille des {\sc pdu}}

\index{PDUSource\_setPDUSizeGenerator}
\begin{verbatim}
void PDUSource_setPDUSizeGenerator(struct PDUSource_t * src,
				   struct randomGenerator_t * rg);
\end{verbatim}

   Ici non plus, pas de surprise. Le premier paramètre est un pointeur
sur la source à modifier et le second est un pointeur sur un
générateur aléatoire à utiliser pour générer les tailles. Les
générateurs aléatoires sont décrits dans la section
\cite{section:rand-gen}.

%........................................................................
%
%........................................................................
\subsubsection{Démarage d'une source}

   Quelle qu'elle soit, une source doit être démarrée afin de produire
des {\sc pdu}. Pour cela, on utilisera la fonction suivante

\index{PDUSource\_start}
\begin{verbatim}
void PDUSource_start(struct PDUSource_t * source);
\end{verbatim}

%........................................................................
%
%........................................................................
\subsubsection{Les sondes}

\index{PDUSource\_addPDUGenerationSizeProbe}
\begin{verbatim}
void PDUSource_addPDUGenerationSizeProbe(struct PDUSource_t * src,
					 struct probe_t *  PDUGenerationSizeProbe);
\end{verbatim}

%-------------------------------------------------------------------------
%
%-------------------------------------------------------------------------
\subsection{Une source périodique}

%-------------------------------------------------------------------------
%
%-------------------------------------------------------------------------
\subsection{Une source {\em Constant Bit Rate}}
