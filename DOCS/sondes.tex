%========================================================================
%
%========================================================================
\section{Les sondes}
\label{section:sondes}

   Les sondes sont implantées dans le modules {\tt probe}. Les sondes
permettent d'enregistrer des mesures de paramètres scalaires. Ce sont
elles qui permettront ensuite d'analyser le résultat de la simulation,
d'évaluer les performances du système, de tracer des courbes, \ldots

   Chaque échantillon d'une mesure prélevé par une sonde est daté. Il
est également possible de n'enregistrer que la date d'un événement,
sans aucune mesure associée.

   Les sondes peuvent être placées en divers points des outils de
simulation via des fonctions de la forme
\lstinline!<type>_add<point>Probe(...)!. Plusieurs sondes peuvent être
ainsi ``chaînées'' sur un même point de mesure (voir plus loin). En
revanche, la même sonde {\bf ne peut pas} être chaînée plusieurs fois.

   Pour chacun des objets décrit dans ce fabuleux document, je
tâcherai de lister (dans une section''Les sondes'') les différents
points de mesure disponibles.

%------------------------------------------------------------------------
%
%------------------------------------------------------------------------
\subsection{Les méthodes de base}

%........................................................................
%
%........................................................................
\subsubsection{Création et destruction}

   Il existe diffférents types de sondes, et chaque type dispose d'au
moins une fonction de création spécificique. Ces différentes fonctions
seront donc listée dans la sous-section suivante.

   La destruction d'une sonde se fait grâce à la fonction 

\index{probe\_delete}
\begin{verbatim}
void probe_delete(struct probe_t * p);
\end{verbatim}

%........................................................................
%
%........................................................................
\subsubsection{L'échantillonage}

   La méthode suivante permet d'échantilloner une valeur dans une
sonde 

\index{probe\_sample}
\begin{verbatim}
void probe_sample(struct probe_t * probe, double value);
\end{verbatim}

   Il est également possible d'échantilloner une date, sans valeur
associée, de la façon suivante 

\index{probe\_sampleEvent}
\begin{verbatim}
void probe_sampleEvent(struct probe_t * probe);
\end{verbatim}

%........................................................................
%
%........................................................................
\subsubsection{Consultation d'un échantillon}

%........................................................................
%
%........................................................................
\subsubsection{La moyenne}

%........................................................................
%
%........................................................................
\subsubsection{Les valeurs extrèmes}

%........................................................................
%
%........................................................................
\subsubsection{La mesure du débit}

   Les valeurs échantillonées pourront souvent être des tailles de
messages, émis ou reçus. Dans ce cas, il peut être intéressant
d'utiliser la fonction suivante qui fournit une mesure du ``débit
instantané''.

\index{probe\_throughput}
\begin{verbatim}
double probe_throughput(struct probe_t * p);
\end{verbatim}

   Bien sûr, le mode d'estimation de cette valeur est dépendant de la
nature de la sonde.
%........................................................................
%
%........................................................................
\subsubsection{La sauvegarde}

   Une sonde peut être dumpée dans un fichier ouvert grâce à la
fonction

{\tt void probe\_graphBarDumpFd(struct probe\_t * probe, int fd, int
format);}

%------------------------------------------------------------------------
%
%------------------------------------------------------------------------
\subsection{Les différents types}

%........................................................................
%
%........................................................................
\subsubsection{La sonde exhaustive}

   La sonde la plus simple est la sonde exhaustive. Avec une telle
sonde, tout échantillon est conservé ``définitivement''. L'avantage
d'une telle technique est que les résultats sont aussi précis qu'on
peut l'espérer, mais l'inconvénient est que cela peut s'avérer très
gourmand en ressources.

   Une sonde exhaustive est créée de la façon suivante

\index{probe\_createExhaustive}
\begin{verbatim}
struct probe_t * probe_createExhaustive();  
\end{verbatim}

   Il n'y a donc aucun paramètre.

%........................................................................
%
%........................................................................
\subsubsection{L'histogramme}

   Une sonde de type histogramme ne conserve que le nombre
d'échantillons par intervalle. Elle est caractérisée par une valeur
minimale, une valeur maximale, et un nombre d'intervalles.

   Une telle sonde est créée grâce à la fonction suivante

\index{probe\_createGraphBar}
\begin{verbatim}
struct probe_t * probe_createGraphBar(double min,
				      double max,
				      unsigned long nbInt);
\end{verbatim}

%
%
%
\paragraph{Normalisation}

   Une sonde de type histograme peut être ``normalisée'', c'est-à-dire
que chaque valeur est divisée par la somme de toutes les valeurs. Si
elle représentait un nombre d'occurence par intervalle, elle
représente donc aprés normalisation un taux d'occurence par
intervalle. 

   Une telle nromalisation se fait à l'aide de la fonction suivante

\index{probe\_graphBarNormalize}
\begin{verbatim}
void probe_graphBarNormalize(struct probe_t * pr);
\end{verbatim}

   Attention au fait qu'une fois normalisée, une sonde n'est plus
modifiable, on ne peut en particulier plus y échantillonner de
nouvelles valeurs.

%
%
%
\paragraph{Création depuis une sonde exhaustive}

%........................................................................
%
%........................................................................
\subsubsection{La fenêtre glissante}

   Ces sondes conservent tous les échantillons sur une fenêtre
glissante dont la taille est fournie en paramêtre du constructeur :

\index{probe\_slidingWindowCreate}
\begin{verbatim}
struct probe_t * probe_slidingWindowCreate(int windowLength);
\end{verbatim}

   A tout moment, une telle sonde conserve donc au maximum les {\tt
windowLength} derniers échantillons. Toutes les valeurs observées
{\em via} une telle sonde (la moyenne, le minimum, le maximum, \ldots)
sont donc obtenues au regard de ces seuls échantillons !

   Si l'on souhaite déterminer des moyennes temporelles, on utilisera
plutôt des sondes par tranche temporelle.

%........................................................................
%
%........................................................................
\subsubsection{La sonde périodique}

   Le but d'une sonde périodique est de prélever un échantillon toutes
les $\tau$ unitées de temps. On se fonde pour cela sur l'idée que la
valeur mesurée n'est modifiée qu'au cours du traitement d'un
événement.

   La cration d'une telle sonde est réalisée par la fonction 

\index{probe\_periodicCreate}
\begin{verbatim}
struct probe_t * probe_periodicCreate(double t);
\end{verbatim}

   Le paramètre {\tt t} est évidemment la période des échantillons.

%........................................................................
%
%........................................................................
\subsubsection{La moyenne mobile}

   Une moyenne du type \lstinline!EMAProbeType! conserve à tout moment
une moyenne mobile calculée à chaque nouvel échantillon $e$ de la façon
suivante $m <- \alpha . m + (1 - \alpha).e$.

%........................................................................
%
%........................................................................
\subsubsection{La moyenne par tranches temporelles}

   Cette sonde conserve une moyenne pour chaque tranche temporelle de
durée {\tt t}, passée en paramètre du constructeur (les fenêtres
temporelles sont sautantes donc disjointes):

\begin{verbatim}
struct probe_t * probe_createTimeAverage(double t);
\end{verbatim}

%------------------------------------------------------------------------
%
%------------------------------------------------------------------------
\subsection{Les ``méta-sondes''}

   À chaque échantillon, certaines sondes mettent à jours des
informations que l'on peut souhaiter conserver (par exemple la moyenne
mobile) afin de tracer leur évolution dans le temps.

   Pour cela, une sonde périodique \lstinline!P! peut être placée sur
une sonde observée \lstinline!O!. La sonde \lstinline!P! permettra
ainsi d'observer une propriété de la sonde\lstinline!O! sur la base
d'un échantillonage à une fréquence caractérisant la sonde
\lstinline!P!. 

   Cette technique permet également de collecter dans une sonde unique
\lstinline!G! des échantillons prélevés dans plusieurs sondes
différentes \lstinline!S1!, \lstinline!S2!, ldots Pour cela,
\lstinline!G! sera ajoutée comme sonde sur les échantillons de
\lstinline!S1!, \lstinline!S2!, ldots 

%------------------------------------------------------------------------
%
%------------------------------------------------------------------------
\subsection{Le chaînage}

   Il peut être parfois intéressant de positionner plusieurs sondes
sur un même point. Cette possibilité est intégrée naturellement
dans{\sc ndes} par la notion de chaînage des sondes.

   Du point de vue de l'utilisateur, chaque fois qu'un module propose
une fonction du type {\tt module\_addUneProbe()}, cette fonction peut
être utilisée pour chacune des sondes à associer. Toutes ces sondes
seront alors traitées à la même enseigne.

   Du point de vue du programmeur, la fonction {\tt
probe\_chain(p1, p2)}\index{probe\_chain}  permet de chaîner la sonde
   {\tt p2} après la sonde {\tt p1} de sorte que chaque appel aux
     fonctions d'échantillonage de la sonde {\tt p1} sera
     automatiquement suivi par le même appel à la sonde {\tt p2}. On
     en trouvera des exemples d'utilisation dans la librairie.

