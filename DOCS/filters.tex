%========================================================================
%
%========================================================================
\section{Les filtres de {\sc pdu}}
\label{section:filtres}

   Un filtre de  {\sc pdu} est un objet très simple permettant de
conditionner des actions au résultat d'une fonction de test appliquée
à une {\sc pdu}.

   Ils sont définis dans {\tt pdu-filter.h} et implantés dans {\tt
pdu-filter.c}. 

   Un exemple d'utilisation des filtres est implanté dans le fichier
{\tt tests.rr-mux.c} qui les utilise pour mesurer le débits de
plusieurs flux multiplexés sans les démultiplexer.

%------------------------------------------------------------------------
%
%------------------------------------------------------------------------
\subsection{Caractérisation et utilisation des filtres}

   Un filtre est essentiellement caractérisé par une fonction de test
et des données privées. Chaque fois que le filtre est appliqué à une
{\sc   pdu}, la fonction de test est invoquée avec comme paramètres
les données privées et la {\sc pdu}. Cette fonction doit alors
retourner 1  ou 0.

%------------------------------------------------------------------------
%
%------------------------------------------------------------------------
\subsection{Déclaration et initialisation}

   Le type d'un filtre est {\tt struct PDUFilter\_t}, et un filtre est
initialisé ainsi :

\index{PDUFilter\_create}
\begin{verbatim}
struct PDUFilter_t * PDUFilter_create();
\end{verbatim}

%------------------------------------------------------------------------
%
%------------------------------------------------------------------------
\subsection{Assignation des paramètres}

\index{PDUFilter\_setPrivate}
\begin{verbatim}
void PDUFilter_setPrivate(struct PDUFilter_t * filter, void * private);
\end{verbatim}

\index{PDUFilter\_setTestFunction}
\begin{verbatim}
void PDUFilter_setTestFunction(struct PDUFilter_t * filter, 
			       int (*test)(void *, struct PDU_t *));
\end{verbatim}

%------------------------------------------------------------------------
%
%------------------------------------------------------------------------
\subsection{Utilisation d'un filtre}

   Lorsqu'une entité a besoin de conditionner une action au résultat
d'un filtre, elle n'a qu'a utiliser la fonction suivante

\index{PDUFilter\_filterPDU}
\begin{verbatim}
int PDUFilter_filterPDU(struct PDUFilter_t * f, struct PDU_t * pdu);
\end{verbatim}
