%========================================================================
%
%========================================================================
\section{Les files de {\sc pdu}}
\label{section:file}

   C'est l'outil de base pour stocker des objets selon une stratégie
{\sc fifo}. Elles sont définies dans le fichier {\tt file\_pdu.h} et
implantées dans le fichier {\tt file\_pdu.c}. Une file ne contient que
des {\sc pdu}s mais, comme on l'a vu, les {\sc pdu}s peuvent être
utilisées pour véhiculer tout et n'importe quoi grâce à leurs données
privées. Il est donc simple de construire des files de n'importe quoi
!

   Par défaut, la taille d'une file n'est limitée quepar les capacités
du système.

%------------------------------------------------------------------------
%
%------------------------------------------------------------------------
\subsection{Création}

   Une file est créée de la façon suivante 

\index{filePDU\_create}
\begin{verbatim}
struct filePDU_t * filePDU_create(void * destination,
			    processPDU_t destProcessPDU);
\end{verbatim}

   Le paramètre \lstinline!destination! est un pointeur sur l'objet
vers lequel sont transmis les objets présents dans la file. Dés qu'un
objet est inséré dans la file, si la destination est disponible, il
lui est envoyé au travers de la fonction \lstinline!destProcessPDU!
passée en paramètre.

%------------------------------------------------------------------------
%
%------------------------------------------------------------------------
\subsection{Insertion}

   On insère dans une file avec une fonction assez classique :

\index{filePDU\_insert}
\begin{verbatim}
void filePDU_insert(struct filePDU_t * file,
		    struct PDU_t * PDU);
\end{verbatim}

   Mais on dispose également de la fonction d'entrée-sortie homogène
avec le reste du simulateur :

\index{filePDU\_processPDU}
\begin{verbatim}
int filePDU_processPDU(void * file,
		       getPDU_t getPDU,
		       void * source);
\end{verbatim}

   Elle permet d'utiliser la file comme n'importe quelle autre entité
dans un simulateur (voir par exemple le premier tutoriel).

%------------------------------------------------------------------------
%
%------------------------------------------------------------------------
\subsection{Extraction}

   L'extraction d'une {\sc pdu} d'une file est réalisé par :

\index{filePDU\_extract}
\begin{verbatim}
struct PDU_t * filePDU_extract(struct filePDU_t * file);
\end{verbatim}

   dont la valeur de retour peut être {\tt NULL} en cas de file vide.

   La aussi, une fonction spécifique à la simulation est fournie :

\index{filePDU\_getPDU}
\begin{verbatim}
struct PDU_t * filePDU_getPDU(void * file);
\end{verbatim}

%------------------------------------------------------------------------
%
%------------------------------------------------------------------------
\subsection{Gestion de la taille}

   Deux fonctions permettent de consulter la taille de la file, la
première donne la taille en nombre de {\sc pdu}s

\index{filePDU\_length}
\begin{verbatim}
int filePDU_length(struct filePDU_t * file);
\end{verbatim}

   la seconde donne la somme cumulée des tailles des {\sc pdu}s
présentes :

\index{filePDU\_size}
\begin{verbatim}
int filePDU_size(struct filePDU_t * file);
\end{verbatim}

%........................................................................
%
%........................................................................
\subsubsection{Taille maximale}

   Il est possible de limiter la taile d'une file. La borne supérieure
peut être exprimée en nombre de {\sc pdu}s présentes ou en taille
cumulée :

\index{filePDU\_setMaxSize}
\index{filePDU\_setMaxLength}
\begin{verbatim}
void filePDU_setMaxSize(struct filePDU_t * file, unsigned long maxSize);
void filePDU_setMaxLength(struct filePDU_t * file, unsigned long maxLength);
\end{verbatim}

   Une valeur nulle signifie l'absence de limite (c'est la valeur par
défaut). Les deux paramètres peuvent être utilisés simultanéments
(c'est alors le plus contraignant qui bloque). Leur valeur peut être
consultée grâce à l'une des fonctions suivantes :

\index{filePDU\_getMaxSize}
\index{filePDU\_getMaxLength}
\begin{verbatim}
unsigned long filePDU_getMaxSize(struct filePDU_t * file);
unsigned long filePDU_getMaxLength(struct filePDU_t * file);
\end{verbatim}

%........................................................................
%
%........................................................................
\subsubsection{Et en cas de débordement ?}

%------------------------------------------------------------------------
%
%------------------------------------------------------------------------
\subsection{Les sondes}

   Les files sont dotées des points de mesure suivants

\begin{description}
   \item[{\tt InsertSize}] pour mesurer la taille des paquets insérés
     dans la file. A chaque insertion d'une {\sc pdu}, la taille de
     cette dernière est échantillonée avec la date d'insertion.
   \item[{\tt ExtractSize}] pour mesurer la taille des paquets extraits
     de la file. A chaque extraction d'une {\sc pdu}, la taille de
     cette dernière est échantillonée avec la date d'extraction.
   \item[{\tt DropSize}]
   \item[{\tt Sejourn}]
\end{description}

   On peut ajouter une sonde sur chacun de ces points de mesure avec
une fonction de la forme

\begin{verbatim}
   void filePDU_add<measure>Probe(struct filePDU_t * file,
                                  struct probe_t   * probe);
\end{verbatim}

