%------------------------------------------------------------------------
%
%------------------------------------------------------------------------
\section{Une source de {\sc pdu}}
\label{section:pdu-source}

   Un objet de type {\tt PDUSource\_t} permet de produire des {\sc
pdu}s selon différentes lois.  Ce type est défini dans le fichier {\tt
     pdu-source.h}. 

   Une source de {\sc pdu}s est essentiellement caractérisée par deux
jeux de parmètres

\begin{itemize}
   \item l'un définissant la loi qui régit les dates de créations
     des {\sc pdu}s ;
   \item l'autre définissant la loi qui régit les tailles de {\sc
     pdu}.
\end{itemize}

%.......................................................................
%
%.......................................................................
\subsection{Création d'une source}

   Une source est créée par la fonction 

\index{PDUSource\_create}
\begin{verbatim}
struct PDUSource_t * PDUSource_create(struct dateGenerator_t * dateGen,
				      void * destination,
				      processPDU_t destProcessPDU);
\end{verbatim}

   Par défaut, la taille des {\sc pdu}s produites est constante égale
à 0.

   Une fonction particulière permet la création d'une source
``déterministe''. Une telle source est initialisée avec une séquence
de couples {\tt \{date, taille\}} qui définit explicitement la liste
des dates et tailles de {\sc pdu}s à produire. Un tel outil ne peut
être utilisé que pour un nombre limité de {\sc pdu}s, mais il est très
pratique à des fins de démonstration ou débogage.

   La fonction d'initialisation est la suivante

\index{PDUSource\_createDeterministic}
\begin{verbatim}
struct PDUSource_t * PDUSource_createDeterministic(struct dateSize * sequence,
						   void * destination,
						   processPDU_t destProcessPDU);
\end{verbatim}

   Le type {\tt struct dateSize} est défini ainsi

\index{struct dateSize}
\begin{verbatim}
struct dateSize {
   double date;
   unsigned int size;
};
\end{verbatim}

   La séquence passée en paramètre à {\tt
 PDUSource\_createDeterministic} doit être ordonnée
chronologiquement et terminée par un couple tel que {\tt \{-1.0, 0\}}
(en fait, il faut simplement que le dernier couple ait une date
antérieur à celle du couple précédent).

%.......................................................................
%
%.......................................................................
\subsection{Activation d'une source}

%.......................................................................
%
%.......................................................................
\subsection{Choix de la loi des dates}

%.......................................................................
%
%.......................................................................
\subsection{Choix de la loi des tailles}

%.......................................................................
%
%.......................................................................
\subsection{Activation d'une source}

%.......................................................................
%
%.......................................................................
\subsection{Les sondes}

%.......................................................................
%
%.......................................................................
\subsection{Quelques sources prédéfinies}

%
%
%
\subsubsection{Une source déterministe}

%
%
%
\subsubsection{Une source périodique}
%
%
%
\subsubsection{Une source {\sc cbr}}
