%========================================================================
%
%========================================================================
\section{Le moteur de simulation}
\label{section:motsim}

%------------------------------------------------------------------------
%
%------------------------------------------------------------------------
\subsection{Gestion du temps}

   Il est possible de demander au simulateur d'essayer de caler
l'évolution du temps simulé sur l'horloge du système.


%------------------------------------------------------------------------
%
%------------------------------------------------------------------------
\subsection{Gestion des événements}

   La gestion des événements est naturellement un point
particulièrement sensible dans un simulateur à événements, même s'il
sont discrets ! {\sc Ndes} se fonde pour le moment sur une vieille
cochonnerie tout droit sortie de mes projets d'étudiant (j'ai été
diplômé, si ça peut vous rassurer). Elle est décrite dans {\tt event.h}
et codée dans {\tt event.c}.

   Les listes d'événements sont également un élément clef, car un
événement est presque toujours présent dans une (et une seule) liste.

%........................................................................
%
%........................................................................
\subsubsection{Caractéristiques principales d'un événement}

   Un événement sert à mettre en \oe{}uvre une action dans la
simulation. Il est donc caractérisé par une fonction et un
pointeur. L'exécution d'un événement consiste alors en l'invocation de
la fonction avec le pointeur comme paramètre.

   Un événement est également caractérisé par une date, mais celle-ci
n'est pas systématiquement significative.

%........................................................................
%
%........................................................................
\subsubsection{Les listes d'événements}

   Les listes d'événements sont les structures dans lesquelles seront
systématiquement placés ces derniers. Pour des raisons de simplicité,
il n'ya qu'un type de liste, mais il y a deux façons de l'utiliser :

\begin{itemize}
   \item Les listes chronologiques sont telles que tous les événements
     contenus sont dans l'ordre croissant des dates.
   \item Dans les listes non triées les événements sont insérés les
     uns après les autres inépendament de leurs dates.
\end{itemize}

   Puisqu'il n'y a qu'un type, c'est au programmeur de veiller à ce
que l'utilisation d'une liste soit cohérente.

   Les listes d'événements sont définies dans {\tt event-list.h} et
implantées dans {\tt event-list.c}.  

%........................................................................
%
%........................................................................
\subsubsection{Création d'un événement}
 
   On peut créer un événement avec la fonction suivante 

\index{event\_create}
\begin{verbatim}
struct event_t * event_create(void (*run)(void *data),
			      void * data);
\end{verbatim}

   Attention, l'événement ne sera exécuté que si il est placé dans la
file d'attente du simulateur. Pour cela, il faut utiliser la fonction
{\tt motSim\_scheduleEvent}. 

%........................................................................
%
%........................................................................
\subsubsection{Insertion d'un événement dans une liste chronologique}

   Un événement peut donc être placé dans une liste chronologique
grâce à la fonction suivante

\index{eventList\_insert}
\begin{verbatim}
void eventList_insert(struct eventList_t * list,
                      struct event_t * event,
                      motSimDate_t date);
\end{verbatim}

   Si la liste passée en paramètre est initialisée et ordonnée de
façon chronologique, alors l'événement \lstinline!event! y est inséré
chronologiquement avec la date \lstinline!date!.

%........................................................................
%
%........................................................................
\subsubsection{Insertion d'un événement dans une liste quelconque}

   Si une liste n'est pas nécessairement chronologique, on peut y
insérer des événements en début de liste ou en fin de liste. Pour
cela, on dispose des deux fonctions suivantes

\index{eventList\_add}
\index{eventList\_append}
\begin{verbatim}
void eventList_add(struct eventList_t * file, struct event_t * event);
void eventList_append(struct eventList_t * list, struct event_t * event);
\end{verbatim}

  La première ajoute l'événement en début de liste, et la seconde
l'ajoute en fin de liste.

%........................................................................
%
%........................................................................
\subsubsection{Création d'un événement périodique}

   Certains événements se reproduisent périodiquement, il est alors
nécessaire d'utiliser, dans le code de traitement de l'événement, la
fonction {\tt event\_add}. Une autre solution consiste à utiliser
l'une des fonctions suivantes 

\index{event\_periodicCreate}
\index{event\_periodicAdd}
\begin{verbatim}
struct event_t * event_PeriodicCreate(void (*run)(void *data),
			      void * data,
			      double date,
		              double period);
void event_periodicAdd(void (*run)(void *data),
		       void * data,
		       double date,
		       double period);

\end{verbatim}

   L'événement sera alors exécuté de façon périodique à partir de la
date fournie en paramètre.

%........................................................................
%
%........................................................................
\subsubsection{Insertion d'un événement dans le simulateur}

   Un événement doit être inséré dans le simulateur afin d'être
finalement exécuté à la date souhaitée. Cette programmation se fait en
insérant l'événement en question dans la liste chronologique des
événements du simulateur.

   Afin de ne pas manipuler directement cette liste, on utilisera la
fonction suivante

\index{motSim\_scheduleEvent}
\begin{verbatim}
void motSim_scheduleEvent(struct event_t * event, motSimDate_t date);
\end{verbatim}

   En fait, cette fonction invoque la fonction
\lstinline!eventList_insert()! avec comme premier paramètre la liste
des événements du simulateur.

   Attention, il est interdit d'insérer un événement situé dans le
passé !

%........................................................................
%
%........................................................................
\subsubsection{Exécution d'un événement ou d'une liste d'événements}

   Un  événement est exécuté au travers de la fontion suivante

\index{event\_run}
\begin{verbatim}
void event_run(struct event_t * event);
\end{verbatim}

   Attention, l'événemenent, une fois exécuté, est détruit.

   Une liste d'événements est exécutée de façon atomique (du point de
vue du système simulé) par la fonction

\index{eventList\_runList}
\begin{verbatim}
void eventList_runList(struct eventList_t * list);
\end{verbatim}

   Les événements sont exécutés dans l'ordre dans lequel ils
apparaissent dans la liste, et ceux indépendament de leur date. Si la
liste a été construite dans l'ordre chronologique, alors ils seront
exécutés dans cet ordre, mais sans lien avec l'horloge du
simulateur. Cette dernière ne progresse donc pas, c'est pour cela que
je dis que cette exécution est atomique : rien d'autre n'évolue dans
le simulateur durant cette exécution.

   Si la liste a été construite avec les fonctions {\tt
eventList\_add} et {\tt evenList\_append}, alors elle sera exécutée
dans l'ordre correspondant.

%........................................................................
%
%........................................................................
\subsubsection{Que dois-je utiliser ?}

   Tout ça a certainement l'air un peu tordu ! J'avoue avoir moi-même
du mal, \ldots Que faut-il donc en retenir ?

   L'utilisateur n'a pas à se préoccuper de tout ça, c'est le premier
point, et il est simple.

   En ce qui concerne le programmeur, il y a en gros deux occasions de
jouer avec les événements.

%
%
%
\paragraph{Créer des événements datés}

   Lorsque l'on a besoin de créer un événement qui devra être exécuté
à une date future connue, on utilisera la fonction {\tt
  motSim\_scheduleEvent} ou la fonction {\tt
  motSim\_scheduleNewEvent}, et on n'a plus à se préoccuper de rien
puisque l'événement sera placé dans la file, exécuté à la bonne date,
puis détruit. 

%
%
%
\paragraph{Créer des événements liés entre eux}

   Dans certaines situations, il peut être nécessaire de déclencher
l'exécution d'un ou plusieurs événement(s) lorsqu'une situation
particulière apparait. Bien sûr, si la date d'apparition de cette
situation est connue, il suffit de créer un ou plusieurs événements
datés.

   Si la date n'est pas connue, on procèdera de la façon suivante.

\begin{enumerate}
   \item On crée une liste d'événements vide grâce à la fonction
     \lstinline!eventList_create!.
   \item On insère dans cette liste chacun des événements qui devront
     être exécutés. On utilise pour cela la fonction
     \lstinline!eventList_append! et il est inutile d'associer une date
     aux événements. 
   \item Lorsque la condition est vérifiée et que les événements
     doivent être exécutés, on utilise la fonction
     \lstinline!eventList_runList!. Les événements seront exécutés
     dans l'ordre de leur insertion dans la file.
\end{enumerate}

%------------------------------------------------------------------------
%
%------------------------------------------------------------------------
\subsection{Gestion mémoire}

   L'allocation de mémoire peut s'avérer problématique, aussi la
fonction suivante a été ajoutée

\index{sim\_malloc}
\begin{verbatim}
void * sim_malloc(int l);
\end{verbatim}

   Elle présente l'avantage de vérifier le résultat de {\tt malloc}
(si les assertions sont activées) et de comptabiliser les appels afins
de détecter une fuite mémoire.

%------------------------------------------------------------------------
%
%------------------------------------------------------------------------
\subsection{Message de débogage}

   La fonction suivante est proposée 

\index{printf\_debug}
\begin{verbatim}
   void printf_debug(int lvl, char * format, ...);
\end{verbatim}

   Elle s'utilise comme {\tt printf} avec en plus un niveau de
débogage (le premier paramètre). Un certain nombre de valeurs sont
définies dans {\tt motsim.h} parmi lesquelles

\index{DEBUG\_ALWAYS}
\begin{verbatim}
#define DEBUG_ALWAYS   0xFFFFFFFF
\end{verbatim}

   qui permet d'assurer que le message sera toujours affiché.

   Les mécanismes de débogage sont activés par la définition de la
macro {\tt DEBUG\_NDES}.

