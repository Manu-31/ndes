%========================================================================
%
%========================================================================
\section{Le moteur de simulation}
\label{section:motsim}

%------------------------------------------------------------------------
%
%------------------------------------------------------------------------
\subsection{Gestion mémoire}

   L'allocation de mémoire peut s'avérer problématique, aussi la
fonction suivante a été ajoutée

\index{sim\_malloc}
\begin{verbatim}
void * sim_malloc(int l);
\end{verbatim}

   Elle présente l'avantage de vérifier le résultat de {\tt malloc}
(si les assertions sont activées) et de comptabiliser les appels afins
de détecter une fuite mémoire.

%------------------------------------------------------------------------
%
%------------------------------------------------------------------------
\subsection{Message de débogage}

   La fonction suivante est proposée 

\index{printf\_debug}
\begin{verbatim}
   void printf_debug(int lvl, char * format, ...);
\end{verbatim}

   Elle s'utilise comme {\tt printf} avec en plus un niveau de
débogage (le premier paramètre). Un certain nombre de valeurs sont
définies dans {\tt motsim.h} parmi lesquelles

\index{DEBUG\_ALWAYS}
\begin{verbatim}
#define DEBUG_ALWAYS   0xFFFFFFFF
\end{verbatim}

   qui permet d'assurer que le message sera toujours affiché.

   Les mécanismes de débogage sont activés par la définition de la
macro {\tt DEBUG\_NDES}.

%------------------------------------------------------------------------
%
%------------------------------------------------------------------------
\subsection{Gestion des événements}

   Elle est décrite dans {\tt event.h} et codée dans {\tt event.c}.

%........................................................................
%
%........................................................................
\subsubsection{Création d'un événement}
 
   On peut créer un événement avec la fonction suivante 

\index{event\_create}
\begin{verbatim}
struct event_t * event_create(void (*run)(void *data),
			      void * data,
			      double date);
\end{verbatim}

   Attention, l'événement ne sera exécuté que si il est placé dans la
file d'attente du simulateur. Pour cela, il faut utiliser la fonction
{\tt motSim\_addEvent}. 

   On peut créer un événement et l'insérer immédiatement dans cette
file grâce à la fonction

\index{event\_add}
\begin{verbatim}
void event_add(void (*run)(void *data),
	       void * data,
	       double date);
\end{verbatim}

%........................................................................
%
%........................................................................
\subsubsection{Création d'un événement périodique}

   Certains événements se reproduisent périodiquement, il est alors
nécessaire d'utiliser, dans le code de traitement de l'événement, la
fonction {\tt event\_add}. Une autre solution consiste à utiliser
l'une des fonctions suivantes 

\index{event\_periodicCreate}
\index{event\_periodicAdd}
\begin{verbatim}
struct event_t * event_PeriodicCreate(void (*run)(void *data),
			      void * data,
			      double date,
		              double period);
void event_periodicAdd(void (*run)(void *data),
		       void * data,
		       double date,
		       double period);

\end{verbatim}

   L'événement sera alors exécuté de façon périodique à partir de la
date fournie en paramètre.


