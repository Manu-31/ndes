%========================================================================
%
%========================================================================
\section{Extension du simulateur}
\label{section:extension}


%------------------------------------------------------------------------
%
%------------------------------------------------------------------------
\subsection{Réalisation d'un ordonnanceur}
\label{subsection:tut-ordo}

   À titre d'exemple, nous allons implanter un ordonnanceur {\em Round
Robin}. Les fichiers source et {\tt Makefile} sont dans le
répertoire {\tt tuto-prog-1}.

%........................................................................
%
%........................................................................
\subsubsection{Caractéristiques de l'ordonnanceur}

   Passons sur les {\tt include} et regardons comment caratériser un
ordonnanceur :

\begin{verbatim}
#define SCHED_RR_NB_INPUT_MAX 8

struct rrSched_t {
   // La destination (typiquement un lien)
   void         * destination;
   processPDU_t   destProcessPDU;

   // Les sources (files d'entrée)
   int        nbSources;
   void     * sources[SCHED_RR_NB_INPUT_MAX];
   getPDU_t   srcGetPDU[SCHED_RR_NB_INPUT_MAX];

   // La dernière source servie par le tourniquet
   int lastServed;
};
\end{verbatim}

   Il nous faut connaître les entités amont, puisque nous voulons les
traiter séparément !

   La fonction de création est assez simple :

\index{rrSched\_create}
\begin{verbatim}
struct rrSched_t * rrSched_create(void * destination,
				  processPDU_t destProcessPDU)
{
   struct rrSched_t * result = (struct rrSched_t * )sim_malloc(sizeof(struct rrSched_t));

   // Gestion de la destination
   result->destination = destination;
   result->destProcessPDU = destProcessPDU;

   // Pas de source définie
   result->nbSources = 0;

   // On commence quelquepart ...
   result->lastServed = 0;

   return result;
}
\end{verbatim}

   On gère la destination, comme toute entité qui peut fournir des
{\sc psu}s doit le faire, et on s'occupe de ses spécificités.

    Une source sera par exemple attribuée de la façon suivante :

\index{rrSched\_addSource}
\begin{verbatim}
void rrSched_addSource(struct rrSched_t * sched,
		       void * source,
		       getPDU_t getPDU)
{
   assert(sched->nbSources < SCHED_RR_NB_INPUT_MAX);

   sched->sources[sched->nbSources] = source;
   sched->srcGetPDU[sched->nbSources++] = getPDU;
}
\end{verbatim}

   Attention, ce n'est pas très robuste ! Mais ce n'est pas l'onjectif
ici, \ldots

%........................................................................
%
%........................................................................
\subsubsection{La fonction {\tt rrSched\_getPDU}}

   C'est cette fonction qui est invoquée par l'entité aval pour
obtenir une {\sc pdu}. C'est dans cette fonction que nous allons donc
implanter l'algorithme d'ordonnancement.

   En voici le code

\index{rrSched\_getPDU}
\begin{verbatim}
struct PDU_t * rrSched_getPDU(void * s)
{
   struct rrSched_t * sched = (struct rrSched_t * )s;
   struct PDU_t * result = NULL;

   assert(sched->nbSources > 0);

   int next = sched->lastServed;

   // Quelle est la prochaine source à servir ?
   do {
      // On cherche depuis la prochaine la première source qui a des
      // choses à nous donner
      next = (next + 1)%sched->nbSources;
      result = sched->srcGetPDU[next](sched->sources[next]);
   } while ((result == NULL) && (next != sched->lastServed));

   if (result)
     sched->lastServed = next;
   return result;
}
\end{verbatim}

   En fait, il n'y a pas grand-chose à en dire ! On applique
l'algorithme d'ordonnancement, et on fournit une {\sc pdu},
éventuellement {\tt NULL} s'il n'y a rien à ordonnancer.

%........................................................................
%
%........................................................................
\subsubsection{La fonction {\tt rrSched\_processPDU}}

   Plus rigolo, passons à la fonction {\tt rrSched\_processPDU} qui
sera invoquée par une source qui dispose d'une {\sc pdu} et qui
souhaite la faire passer à l'ordonnanceur.

   Celui-ci devra la traiter s'il le peut, mais ne pas la prendre s'il
ne peut pas la traiter. Voici le début de la fonction

\index{rrSched\_processPDU}
\begin{verbatim}
int rrSched_processPDU(void *s,
		       getPDU_t getPDU,
		       void * source)
{
   int result;
   struct rrSched_t * sched = (struct rrSched_t *)s;

   printf_debug(DEBUG_SCHED, "in\n");
\end{verbatim}

   J'ai utiliser la fonction de débogage à titre d'exemple. Il y en a
d'autres dans le fichier source. On se contente ici d'un {\em cast}
pour respecter le prototype des fonctions d'échange.

   Il suffit maintenant de demander à l'entité aval de venir chercher
une {\sc pdu} (celle-là ou une autre, c'est l'algo qui le dira,
mais on ne voit pas comment ça pourrait en être une autre~!).

\begin{verbatim}
   // On dit à l'aval de venir chercher une PDU, ce
   // qui déclanchera l'ordonnancement
   result = sched->destProcessPDU(sched->destination, rrSched_getPDU, sched);
\end{verbatim}

   On oublie pas de renvoyer le résultat !

\begin{verbatim}

   printf_debug(DEBUG_SCHED, "out\n");
   return result;
}
\end{verbatim}

%........................................................................
%
%........................................................................
\subsubsection{Utilisation}

   Ben ça c'est pas m'échant, je te laisse lire les sources !

